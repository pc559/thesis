%\newcommand{\eps}{\epsilon}
\newcommand{\Pmax}{p_\text{max}} 	
\newcommand{\Hint}{H_{int}}
\newcommand{\kbar}{\bar{k}}
\newcommand{\Delk}{\Delta}
\newcommand{\Sumk}{\Sigma}
\newcommand{\Qrot}{Q_{pq}(\tau_s)}
\newcommand{\shapecor}{\mathcal{S}}
\newcommand{\ampcor}{\mathcal{A}}
\newcommand{\totalcor}{\mathcal{E}}
\newcommand{\threeLs}{L_p(\kbar_1)L_q(\kbar_2)L_r(\kbar_3)}
\newcommand{\threePs}{P_p(\kbar_1)P_q(\kbar_2)P_r(\kbar_3)}
\newcommand{\threeQs}{Q_p(k_1)Q_q(k_2)Q_r(k_3)}
\newcommand{\Lbasic}{\mathcal{P}_0}
\newcommand{\Linvk}{\mathcal{P}_1}
\newcommand{\Lnsinv}{\mathcal{P}^{n_s}_1}
\newcommand{\Lnsboth}{\mathcal{P}^{n_s}_{01}}
\newcommand{\Linvksq}{\mathcal{P}_2}
\newcommand{\Lns}{\mathcal{P}^{n_s}_2}
\newcommand{\Fbasic}{\mathcal{F}_0}
\newcommand{\Finvk}{\mathcal{F}_1}
\newcommand{\Finvksq}{\mathcal{F}_2}
\newcommand{\quadpot}{V_{\phi^2}(\phi)}
%\newcommand{\threeqs}{q_p(\kbar_1)\,q_r(\kbar_2)\,q_s(\kbar_3)}
\newcommand{\threeqs}{q_p(k_1)\,q_r(k_2)\,q_s(k_3)}
\newcommand{\threeqstilde}{q_{\tilde{p}}(k_1)\,q_{\tilde{r}}(k_2)\,q_{\tilde{s}}(k_3)}
\newcommand{\kmin}{{k_\text{min}}}
\newcommand{\kmax}{{k_\text{max}}}
\newcommand{\fnl}{f_{NL}}
\newcommand{\fnllocal}{f^{local}_{NL}}
\newcommand{\fnlequil}{f^{equil}_{NL}}
\newcommand{\fnlortho}{f^{ortho}_{NL}}

\chapter{Introduction to cosmology/inflation}
\section{General intro}\label{sec:general_intro}
    \subsection{GR.}
    \subsection{$\Lambda CDM$.}
\section{Before the Big Bang (cf popsci)}
    \subsection{Motivate inflation.}
    \subsection{Define SR params.}
\section{Determining the weighting of a coin}
    \subsection{The PS.}
\section{Simons, Planck}
    \subsection{Descriptions}
\section{Goals}
    \subsection{connecting inflation models directly to observations, through the bispectrum ("which will be defined in section...")}
    \subsection{Constraining the parameters of inflation models, not pheno templates ($f_{NL}$ etc).}
    \subsection{Full shape information, not point samples or a limit.}
    \subsection{Efficient numerics gives access to more accurate, and in some cases new, feature shapes.}
\section{Methods}
    \subsection{building separability into the tree-level in-in formalism}
    \subsection{CMB calculation expensive, but need only be done once per primordial basis.}
    \subsection{So, want a basis expansion that converges quickly for a broad range of inflation models.}
    \subsection{Convergence on the cube is different to the tetrapyd.}
    \subsection{Turns out to be much faster at primordial level than previous numerical methods.}
\section{Results}
    \subsection{First development/implementation of the formalism for high orders, features.}
    \subsection{Pointed out the central issue of the cube vs tetra problem.}
    \subsection{Found a basis with broad descriptive power.}
    \subsection{Validated these methods on interesting examples.}
    \subsection{Explore and characterise DBI reso model TBC?? Validity of approximations, eg Tanh kink.}
    \subsection{Connect to CMB, get constraints TBC??}

