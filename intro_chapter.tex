%\newcommand{\eps}{\epsilon}
\newcommand{\Pmax}{p_\text{max}} 	
\newcommand{\Hint}{H_{int}}
\newcommand{\kbar}{\bar{k}}
\newcommand{\Delk}{\Delta}
\newcommand{\Sumk}{\Sigma}
\newcommand{\Qrot}{Q_{pq}(\tau_s)}
\newcommand{\shapecor}{\mathcal{S}}
\newcommand{\ampcor}{\mathcal{A}}
\newcommand{\totalcor}{\mathcal{E}}
\newcommand{\threeLs}{L_p(\kbar_1)L_q(\kbar_2)L_r(\kbar_3)}
\newcommand{\threePs}{P_p(\kbar_1)P_q(\kbar_2)P_r(\kbar_3)}
\newcommand{\threeQs}{Q_p(k_1)Q_q(k_2)Q_r(k_3)}
\newcommand{\Lbasic}{\mathcal{P}_0}
\newcommand{\Linvk}{\mathcal{P}_1}
\newcommand{\Lnsinv}{\mathcal{P}^{n_s}_1}
\newcommand{\Lnsboth}{\mathcal{P}^{n_s}_{01}}
\newcommand{\Linvksq}{\mathcal{P}_2}
\newcommand{\Lns}{\mathcal{P}^{n_s}_2}
\newcommand{\Fbasic}{\mathcal{F}_0}
\newcommand{\Finvk}{\mathcal{F}_1}
\newcommand{\Finvksq}{\mathcal{F}_2}
\newcommand{\quadpot}{V_{\phi^2}(\phi)}
%\newcommand{\threeqs}{q_p(\kbar_1)\,q_r(\kbar_2)\,q_s(\kbar_3)}
\newcommand{\threeqs}{q_p(k_1)\,q_r(k_2)\,q_s(k_3)}
\newcommand{\threeqstilde}{q_{\tilde{p}}(k_1)\,q_{\tilde{r}}(k_2)\,q_{\tilde{s}}(k_3)}
\newcommand{\kmin}{{k_\text{min}}}
\newcommand{\kmax}{{k_\text{max}}}
\newcommand{\fnl}{f_{NL}}
\newcommand{\fnllocal}{f^{local}_{NL}}
\newcommand{\fnlequil}{f^{equil}_{NL}}
\newcommand{\fnlortho}{f^{ortho}_{NL}}

\chapter{Introduction to cosmology}\label{chapter:intro_general}
We outline a formalism, do quantitative comparisons, describe difficulties encountered$\ldots$

Calculating the primordial bispectrum predicted by a model of inflation and
comparing it to what we see in the sky is very computationally intensive, necessitating layers
of approximations and limiting the models which can be constrained.  Exploiting the inherent separability
of the tree level in-in formalism using expansions in separable basis functions provides a means by which to
obviate some of these difficulties. Here, we develop this approach further into a practical and efficient
numerical methodology which can be applied to a much wider and more complicated range of bispectrum
phenomenology, making an important step forward towards observational pipelines which can directly
confront specific models of inflation.  We describe a simple augmented Legendre polynomial basis
and its advantages, then test the method on single-field inflation models with non-trivial phenomenology,
showing that our calculation of these coefficients is fast and accurate to high orders.

\section{General intro}\label{sec:general_intro}
%\subsection{The bispectrum}
The primordial bispectrum is one of the main
characteristics used to distinguish between models of inflation. While it is well
known that the physics of inflation must have been extremely close
to linear, and the initial seeds of structure it laid down
very close to Gaussian, there is expected to have been some level of coupling
between the Fourier modes of the perturbations.
In the simplest example of an inflation model this is
expected to be unobservable~\cite{Maldacena},
but the possibility remains that inflation was driven by
more complex physics that may have left an observable imprint on our universe today.
Some models of inflation have interactions that predict non-Gaussian
correlations at observable levels. Ways this can happen include
self-interactions~\cite{px_burrage,dbi_in_the_sky},
interactions between multiple fields~\cite{Byrnes_2010},
sharp features~\cite{adshead}
and periodic features~\cite{flauger_pajer_resonant}.
However, constraining such imprints is extremely difficult observationally.
Even once the data has been obtained, using existing methods it is
extremely computationally intensive to translate this into constraints
on specific inflation scenarios. Much progress has been made by course-graining
the model space into a small number of approximate templates,
and leveraging the simplifying characteristic of separability
with respect to the three parameters of the bispectrum~\cite{Komatsu_2005, Munchmeyer_2014}.


The primordial bispectrum is the Fourier equivalent of the
three-point correlator of the primordial curvature perturbation.
If this field is Gaussian, the bispectrum vanishes, so
it is a valuable measure of the interactions in play during inflation.
If some inflation model predicts a bispectrum that is sufficiently well approximated by
the standard separable templates, the constraints on those standard templates
can be translated into constraints on the parameters of the model.
The fact that all primordial templates estimated thus far from the CMB
are consistent with zero has already provided such constraints
in certain scenarios~\cite{Planck_NG_2015, Planck_NG_2018}.
With this high-precision \textit{Planck} data, and data from forthcoming experiments
such as the Simons Observatory (SO)~\cite{simons}
and CMB-S4~\cite{abazajian2016cmbs4},
robust pipelines must be developed to circumvent the computational difficulties and
extract the maximum amount of information possible.
Due to the nature of bispectrum estimation in the CMB and
LSS~\cite{lss_baldauf,lss_karagiannis,chen_future_lss,Scoccimarro_2012}
constraining an arbitrary template is difficult.
Our aim in this work is to develop the inflationary part
of a pipeline to allow to efficiently test a much broader range of models.
In this work, we explore shapes arising from tree-level effects in single field models.
We do this numerically, allowing quantitative results for a broad
range of models, and avoiding extra approximations.
Our general aim is to apply the modal philosophy of~\cite{FergShell_1,FergShell_2,FergShell_3}
to calculating primordial bispectra.
This modal philosophy is a flexible method that has broadened the range of constrained
bispectrum templates, by expanding them in a carefully chosen basis.
The Modal estimator is thus capable of constraining
non-separable templates, while the KSW estimator cannot.
In this work we exploit the intrinsic separability of the
tree-level in-in formalism to apply these methods at the level of inflation.
Expressing the primordial bispectrum in a separable
basis expansion leads to vast increases in efficiency both at the primordial
and late-universe parts of the calculation.
The main advantage is that expressing the primordial shape function
in this way reduces the process of bispectrum estimation in the CMB to a
cost which is large, but need only be paid once per basis,
not per scenario.
A proof of concept of this approach at the primordial level was presented in~\cite{Funakoshi},
and the details of the bispectrum estimation part will be detailed in~\cite{Sohn_2021}.
We go beyond the work of~\cite{Funakoshi} both in developing the choice of basis
(the feasibility of the method depending vitally on the chosen basis
achieving sufficiently fast convergence in a broad range of interesting models)
and in the methods we use to allow us to go to much higher order in our modal expansion,
allowing us to apply the method to feature bispectra for the first time.


The paper is organised as follows. In chapter~\ref{chapter:intro_bispectra} we present brief reviews
of the various parts of the pipeline that connects inflation scenarios to observations
through the bispectrum.
We review the usual paradigm of bispectrum estimation in the CMB,
and the motivation for separable bispectra. We review the in-in formalism,
for calculating the tree level bispectrum for a given model of inflation.
We review $P(X,\phi)$ models of inflation as an example, and
some of the usual approximate bispectrum templates
that we aim to bypass.
We will draw our validation scenarios from these models.
We discuss previous numerical codes for
calculating the primordial bispectrum $k$-configuration by $k$-configuration,
which contrasts our separable basis expansion.
We review the previous work in achieving separability through modal expansions
in~\cite{Funakoshi},
and we discuss methods of testing
numerical bispectrum results, defining our relative difference measurement.
In chapter~\ref{chapter:methods} we present our methods.
Since the paradigm we aim to present is only viable if we can find a basis
that can efficiently represent a wide variety of bispectra,
we begin with this distinct and separate, but nonetheless vital, discussion.
We discuss the effects of the
non-physical $k$-configurations on the convergence of our expansion on
the tetrapyd, and present an efficient basis.
Then, we recast the usual in-in calculation into an explicitly separable form,
in terms of an expansion in an arbitrary basis,
and detail our methods for carefully calculating the coefficients to high order.
In section~\ref{sec:validation} we validate our methods and implementation
on inflation scenarios with varied features from the literature,
and we finish with a discussion of future work in chapter~\ref{chapter:conclusion}.
    \subsection{Fundamentals}
    Words
\newpage
    Words
\newpage
    \subsection{$\lcdm$}
    Words
\newpage
    Words
\newpage
    Words
\newpage
    Words
\newpage
\section{Starting conditions for $\lcdm$}
    \subsection{Motivations for inflation (before the Big Bang?)}
    Words
\newpage
    Words
\newpage
    Words
\newpage
    \subsection{Criteria for successful inflation}
    Words
\newpage
    Words
\newpage
\section{Statistical observables}
    \subsection{Determining the weighting of a coin}
    The technicalities and limitations involved with determining a statistical observable.
\newpage
    \subsection{Power spectra}
    Words
\newpage
\section{Observational data}
    \subsection{Simons, Planck}
    Words
\newpage
    \subsection{Future missions}
    Words
\newpage
\section{Goals}
    \subsection{connecting inflation models directly to observations, through the bispectrum ("which will be defined in section...")}
    Words
    \subsection{Constraining the parameters of inflation models, not pheno templates ($f_{NL}$ etc).}
    Words
    \subsection{Full shape information, not point samples or a limit.}
    Words
    \subsection{Efficient numerics gives access to more accurate, and in some cases new, feature shapes.}
    Words
\newpage
\section{Methods}
    \subsection{building separability into the tree-level in-in formalism}
    Words
    \subsection{CMB calculation expensive, but need only be done once per primordial basis.}
    Words
    \subsection{So, want a basis expansion that converges quickly for a broad range of inflation models.}
    Words
    \subsection{Convergence on the cube is different to the tetrapyd.}
    Words
    \subsection{Turns out to be much faster at primordial level than previous numerical methods.}
    Words
\newpage
\section{Results}
    \subsection{First development/implementation of the formalism for high orders, features.}
    We recognised\ldots which allowed us to\ldots
    \subsection{Pointed out the central issue of the cube vs tetra problem.}
    Words
    \subsection{Found a basis with broad descriptive power.}
    Words
    \subsection{Validated these methods on interesting examples.}
    Words
    \subsection{Explore and characterise DBI reso model TBC? Validity of approximations, eg Tanh kink.}
    Words
    \subsection{Connect to CMB, get constraints TBC?}
    Words

