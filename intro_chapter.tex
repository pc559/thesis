%\newcommand{\delphi}{\delta\phi}
\newcommand{\udot}{\dot{u}}
\renewcommand{\Re}{\operatorname{Re}}
\renewcommand{\Im}{\operatorname{Im}}
\newcommand{\dzeta}{\dot{\zeta}}
\newcommand{\ahat}{\hat{a}}
\newcommand{\epscs}{\varepsilon_{s}}
\newcommand{\vecx}{\mathbf{x}}
\newcommand{\veck}{\mathbf{k}}
\newcommand{\nhatbrkt}{\left(\hat{\mathbf{n}}\right)}
\newcommand{\primpowerspec}{\mathcal{P}_{\mathcal{R}}}
\newcommand{\triplea}{a_{l_1m_1}a_{l_2m_2}a_{l_3m_3}}
\newcommand{\tripleaobs}{a^{obs}_{l_1m_1}a^{obs}_{l_2m_2}a^{obs}_{l_3m_3}}
\newcommand{\bk}{\mathbf{k}}
\newcommand{\totalcortet}{\mathcal{E}^{tetra}}
\newcommand{\totalcorcub}{\mathcal{E}^{cube}}
\newcommand{\primodal}{{\textit{\textsc{Primodal}}}}
\newcommand{\cmbbest}{{\textit{\textsc{CMB-BEst}}}}
\newcommand{\planck}{\textit{Planck}}
\newcommand{\lcdm}{\Lambda CDM}
\newcommand{\cmb}{\text{CMB}}
\newcommand{\eps}{\epsilon}
\newcommand{\Pmax}{p_\text{max}} 	
\newcommand{\Hint}{H_{int}}
\newcommand{\Lint}{L_{int}}
\newcommand{\kbar}{\bar{k}}
\newcommand{\Delk}{\Delta}
\newcommand{\Sumk}{\Sigma}
\newcommand{\Qrot}{Q_{pq}(\tau_s)}
\newcommand{\shapecor}{\mathcal{S}}
\newcommand{\ampcor}{\mathcal{A}}
\newcommand{\totalcor}{\mathcal{E}}
\newcommand{\threeLs}{L_p(\kbar_1)L_q(\kbar_2)L_r(\kbar_3)}
\newcommand{\threePs}{P_p(\kbar_1)P_q(\kbar_2)P_r(\kbar_3)}
\newcommand{\threeQs}{Q_p(k_1)Q_q(k_2)Q_r(k_3)}
\newcommand{\Lbasic}{\mathcal{P}_0}
\newcommand{\Linvk}{\mathcal{P}_1}
\newcommand{\Lnsinv}{\mathcal{P}^{n_s}_1}
\newcommand{\Lnsboth}{\mathcal{P}^{n_s}_{01}}
\newcommand{\scalingbasis}{\textit{scaling}}
\newcommand{\resobasis}{\textit{resonant}}
\newcommand{\Linvksq}{\mathcal{P}_2}
\newcommand{\Lns}{\mathcal{P}^{n_s}_2}
\newcommand{\Fbasic}{\mathcal{F}_0}
\newcommand{\Finvk}{\mathcal{F}_1}
\newcommand{\Finvksq}{\mathcal{F}_2}
\newcommand{\quadpot}{V_{\phi^2}(\phi)}
%\newcommand{\threeqs}{q_p(\kbar_1)\,q_r(\kbar_2)\,q_s(\kbar_3)}
\newcommand{\threeqs}{q_p(k_1)\,q_r(k_2)\,q_s(k_3)}
\newcommand{\threeqstilde}{q_{\tilde{p}}(k_1)\,q_{\tilde{r}}(k_2)\,q_{\tilde{s}}(k_3)}
\newcommand{\kmin}{{k_\text{min}}}
\newcommand{\kmax}{{k_\text{max}}}
\newcommand{\fnl}{f_{NL}}
\newcommand{\fnllocal}{f^{local}_{NL}}
\newcommand{\fnlequil}{f^{equil}_{NL}}
\newcommand{\fnlortho}{f^{ortho}_{NL}}

\chapter{Introduction to cosmology}\label{chapter:intro_general}
\section{General introduction}\label{sec:general_intro}
%\subsection{The bispectrum}
(Move the below to intro2.)
The primordial bispectrum is one of the main
characteristics used to distinguish between models of inflation. While it is well
known that the physics of inflation must have been extremely close
to linear, and the initial seeds of structure it laid down
very close to Gaussian, there is expected to have been some level of coupling
between the Fourier modes of the perturbations.
In the simplest example of an inflation model this is
expected to be unobservable~\cite{Maldacena},
but the possibility remains that inflation was driven by
more complex physics that may have left an observable imprint on our universe today.
Some models of inflation have interactions that predict non-Gaussian
correlations at observable levels. Ways this can happen include
self-interactions~\cite{px_burrage,dbi_in_the_sky},
interactions between multiple fields~\cite{Byrnes_2010},
sharp features~\cite{adshead}
and periodic features~\cite{flauger_pajer_resonant}.
However, constraining such imprints is extremely difficult observationally.
Even once the data has been obtained, using existing methods it is
extremely computationally intensive to translate this into constraints
on specific inflation scenarios. Much progress has been made by course-graining
the model space into a small number of approximate templates,
and leveraging the simplifying characteristic of separability
with respect to the three parameters of the bispectrum~\cite{Komatsu_2005, Munchmeyer_2014}.


The primordial bispectrum is the Fourier equivalent of the
three-point correlator of the primordial curvature perturbation.
If this field is Gaussian, the bispectrum vanishes, so
it is a valuable measure of the interactions in play during inflation.
If some inflation model predicts a bispectrum that is sufficiently well approximated by
the standard separable templates, the constraints on those standard templates
can be translated into constraints on the parameters of the model.
The fact that all primordial templates estimated thus far from the CMB
are consistent with zero has already provided such constraints
in certain scenarios~\cite{Planck_NG_2015, Planck_NG_2018}.
With this high-precision \planck~data, and data from forthcoming experiments
such as the Simons Observatory (SO)~\cite{simons}
and CMB-S4~\cite{abazajian2016cmbs4},
robust pipelines must be developed to circumvent the computational difficulties and
extract the maximum amount of information possible.
Due to the nature of bispectrum estimation in the CMB and
LSS~\cite{lss_baldauf,lss_karagiannis,chen_future_lss,Scoccimarro_2012}
constraining an arbitrary template is difficult.
Our aim in this work is to develop the inflationary part
of a pipeline to allow to efficiently test a much broader range of models.
In this work, we explore shapes arising from tree-level effects in single field models.
We do this numerically, allowing quantitative results for a broad
range of models, and avoiding extra approximations.
Our general aim is to apply the modal philosophy of~\cite{FergShell_1,FergShell_2,FergShell_3}
to calculating primordial bispectra.
This modal philosophy is a flexible method that has broadened the range of constrained
bispectrum templates, by expanding them in a carefully chosen basis.
The Modal estimator is thus capable of constraining
non-separable templates, while the KSW estimator cannot.
In this work we exploit the intrinsic separability of the
tree-level in-in formalism to apply these methods at the level of inflation.
Expressing the primordial bispectrum in a separable
basis expansion leads to vast increases in efficiency both at the primordial
and late-universe parts of the calculation.
The main advantage is that expressing the primordial shape function
in this way reduces the process of bispectrum estimation in the CMB to a
cost which is large, but need only be paid once per basis,
not per scenario.
A proof of concept of this approach at the primordial level was presented in~\cite{Funakoshi},
and the details of the bispectrum estimation part will be detailed in~\cite{Sohn_2021}.
We go beyond the work of~\cite{Funakoshi} both in developing the choice of basis
(the feasibility of the method depending vitally on the chosen basis
achieving sufficiently fast convergence in a broad range of interesting models)
and in the methods we use to allow us to go to much higher order in our modal expansion,
allowing us to apply the method to feature bispectra for the first time.


The paper is organised as follows. In chapter~\ref{chapter:intro_bispectra} we present brief reviews
of the various parts of the pipeline that connects inflation scenarios to observations
through the bispectrum.
We review the usual paradigm of bispectrum estimation in the CMB,
and the motivation for separable bispectra. We review the in-in formalism,
for calculating the tree level bispectrum for a given model of inflation.
We review $P(X,\phi)$ models of inflation as an example, and
some of the usual approximate bispectrum templates
that we aim to bypass.
We will draw our validation scenarios from these models.
We discuss previous numerical codes for
calculating the primordial bispectrum $k$-configuration by $k$-configuration,
which contrasts our separable basis expansion.
We review the previous work in achieving separability through modal expansions
in~\cite{Funakoshi},
and we discuss methods of testing
numerical bispectrum results, defining our relative difference measurement.
In chapter~\ref{chapter:methods} we present our methods.
Since the paradigm we aim to present is only viable if we can find a basis
that can efficiently represent a wide variety of bispectra,
we begin with this distinct and separate, but nonetheless vital, discussion.
We discuss the effects of the
non-physical $k$-configurations on the convergence of our expansion on
the tetrapyd, and present an efficient basis.
Then, we recast the usual in-in calculation into an explicitly separable form,
in terms of an expansion in an arbitrary basis,
and detail our methods for carefully calculating the coefficients to high order.
In section~\ref{sec:validation} we validate our methods and implementation
on inflation scenarios with varied features from the literature,
and we finish with a discussion of future work in chapter~\ref{chapter:conclusion}.
    \subsection{Fundamentals}
    (A discussion of GR etc.)
\newpage
    Words
\newpage
    \subsection{$\lcdm$}
    (Outlining $\lcdm$.)
\newpage
    Words
\newpage
    Words
\newpage
    Words
\newpage
\section{Initial conditions for $\lcdm$}
    \subsection{Motivations for inflation}
    Before the popsci Big Bang. Horizon problem etc.
\newpage
    Words
\newpage
    Words
\newpage
    \subsection{Criteria for successful inflation}
    Sufficient e-folds etc.
\newpage
    Words
\newpage
\section{Statistical observables}
    \subsection{Checking dice for fairness}
    The technicalities and limitations involved in determining a statistical observable,
    the concept of variance, cosmic variance.
    Ergodicity: ``In an ergodic scenario, the average outcome of the group is the same as the average outcome of the individual over time. An example of an ergodic systems would be the outcomes of a coin toss (heads/tails). If 100 people flip a coin once or 1 person flips a coin 100 times, you get the same outcome.'' from https://taylorpearson.me/ergodicity/
\newpage
    \subsection{Power spectra}
    Define n-point correlations, their Fourier transforms, talk about them as observables.
\newpage
\section{Observational data}
    \subsection{\planck, Simons} 
    High-level descriptions.
\newpage
    \subsection{Future missions}
    High-level descriptions.
\newpage
\section{Outline of thesis}
    \subsection{Goals}
    \begin{enumerate}
        \item Connecting inflation models directly to observations,
            through the bispectrum.
        \item Constraining the parameters of inflation models, not phenomenological templates and $f_{NL}$.
        \item To obtain the full shape information, not point samples or a limit.
        \item Efficient numerics gives access to more accurate, and in some cases new, feature shapes.
    \end{enumerate}
\newpage
    \subsection{Methods}
    \begin{enumerate}
        \item Building separability into the tree-level in-in formalism.
        \item The CMB calculation~\cite{Sohn_2021}: expensive, but need only be done once per primordial basis.
        \item So, we want a basis expansion that converges quickly for a broad range of inflation models.
        \item Convergence on the cube is different to the tetrapyd.
        \item Turns out to be much faster at primordial level than previous numerical methods
            (as it in a sense converges way faster, and as it enables us to use faster numerical methods than otherwise).
    \end{enumerate}
\newpage
    \subsection{Results}
    \begin{enumerate}
        \item First development/implementation of the formalism for calculating the expansion to high orders.
        \item We recognised and described the central issue of the cube vs tetra problem.
        \item Found a basis with broad descriptive power (and other less powerful basis sets).
        \item This allowed the first validation of these methods on features.
        \item Explore and characterise DBI reso model TBC? Validity of approximations, eg Tanh kink.
        \item Connect to CMB, get constraints TBC?
    \end{enumerate}
