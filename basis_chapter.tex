%\newcommand{\delphi}{\delta\phi}
\newcommand{\udot}{\dot{u}}
\renewcommand{\Re}{\operatorname{Re}}
\renewcommand{\Im}{\operatorname{Im}}
\newcommand{\dzeta}{\dot{\zeta}}
\newcommand{\ahat}{\hat{a}}
\newcommand{\epscs}{\varepsilon_{s}}
\newcommand{\vecx}{\mathbf{x}}
\newcommand{\veck}{\mathbf{k}}
\newcommand{\nhatbrkt}{\left(\hat{\mathbf{n}}\right)}
\newcommand{\primpowerspec}{\mathcal{P}_{\mathcal{R}}}
\newcommand{\triplea}{a_{l_1m_1}a_{l_2m_2}a_{l_3m_3}}
\newcommand{\tripleaobs}{a^{obs}_{l_1m_1}a^{obs}_{l_2m_2}a^{obs}_{l_3m_3}}
\newcommand{\bk}{\mathbf{k}}
\newcommand{\totalcortet}{\mathcal{E}^{tetra}}
\newcommand{\totalcorcub}{\mathcal{E}^{cube}}
\newcommand{\primodal}{{\textit{\textsc{Primodal}}}}
\newcommand{\cmbbest}{{\textit{\textsc{CMB-BEst}}}}
\newcommand{\planck}{\textit{Planck}}
\newcommand{\lcdm}{\Lambda CDM}
\newcommand{\cmb}{\text{CMB}}
\newcommand{\eps}{\epsilon}
\newcommand{\Pmax}{p_\text{max}} 	
\newcommand{\Hint}{H_{int}}
\newcommand{\Lint}{L_{int}}
\newcommand{\kbar}{\bar{k}}
\newcommand{\Delk}{\Delta}
\newcommand{\Sumk}{\Sigma}
\newcommand{\Qrot}{Q_{pq}(\tau_s)}
\newcommand{\shapecor}{\mathcal{S}}
\newcommand{\ampcor}{\mathcal{A}}
\newcommand{\totalcor}{\mathcal{E}}
\newcommand{\threeLs}{L_p(\kbar_1)L_q(\kbar_2)L_r(\kbar_3)}
\newcommand{\threePs}{P_p(\kbar_1)P_q(\kbar_2)P_r(\kbar_3)}
\newcommand{\threeQs}{Q_p(k_1)Q_q(k_2)Q_r(k_3)}
\newcommand{\Lbasic}{\mathcal{P}_0}
\newcommand{\Linvk}{\mathcal{P}_1}
\newcommand{\Lnsinv}{\mathcal{P}^{n_s}_1}
\newcommand{\Lnsboth}{\mathcal{P}^{n_s}_{01}}
\newcommand{\scalingbasis}{\textit{scaling}}
\newcommand{\resobasis}{\textit{resonant}}
\newcommand{\Linvksq}{\mathcal{P}_2}
\newcommand{\Lns}{\mathcal{P}^{n_s}_2}
\newcommand{\Fbasic}{\mathcal{F}_0}
\newcommand{\Finvk}{\mathcal{F}_1}
\newcommand{\Finvksq}{\mathcal{F}_2}
\newcommand{\quadpot}{V_{\phi^2}(\phi)}
%\newcommand{\threeqs}{q_p(\kbar_1)\,q_r(\kbar_2)\,q_s(\kbar_3)}
\newcommand{\threeqs}{q_p(k_1)\,q_r(k_2)\,q_s(k_3)}
\newcommand{\threeqstilde}{q_{\tilde{p}}(k_1)\,q_{\tilde{r}}(k_2)\,q_{\tilde{s}}(k_3)}
\newcommand{\kmin}{{k_\text{min}}}
\newcommand{\kmax}{{k_\text{max}}}
\newcommand{\fnl}{f_{NL}}
\newcommand{\fnllocal}{f^{local}_{NL}}
\newcommand{\fnlequil}{f^{equil}_{NL}}
\newcommand{\fnlortho}{f^{ortho}_{NL}}

\chapter{Decomposing primordial shapes}
\section{Pulling out the $k$-dependance}
    \subsection{Set up formalism.}
    \subsection{Forced to decompose on the cube.}
    \subsection{Defining the notation for the two main time integrals.}
\section{Inner product choice}
    \subsection{Easier to interpret, more stringent.}
    \subsection{Notes on observable convergence being the deciding factor.}
\section{Template testing}
    \subsection{Need to understand expected convergence before validation on numerical results.}
\section{Decomposing shapes on the cube vs tetrapyd}
    \subsection{Large non-physical contributions.}
    \subsection{FIGURE: DBI on cube vs tetra.}
\section{Setting up a basis: augmentation}
    \subsection{Using modified GS.}
    \subsection{Legendre and Fourier as building blocks}
\section{Basis choice matters!}
    \subsection{Why the basic basis expansion is so bad.}
    \subsection{My new basis sets, and their dramatic improvement.}
    \subsection{Nice table with descriptions and some single-number comparison on examples.}
    \subsection{FIGURES: Recon for Malda, DBI, scale-inv, with P0, P1, F0, F1 vs $P_{max}$.}
    \subsection{FIGURES: Recon for cos, cosDBI, with P0, P1, F0, F1 vs $P_{max}$.}
    \subsection{FIGURES: Recon for cos-log, cos-logDBI, with P0, P1, F0, F1 vs $P_{max}$.}
    \subsection{FIGURES: Recon for scaled-DBI, with P0, P1, P1ns, P01ns, $P_log$ vs $P_{max}$.}
\section{Not obvious, but $k_{ratio}$ matters!}
\section{The $P_{max}$-$k_{ratio}$ tradeoff (and DBI vs Equil).}
\section{Log basis for basis without PS info.}
\section{Loginv basis for resonant shapes.}
\section{Factor basis?}
\section{Conclusion}
    \subsection{Basic Legendres/Fourier are not sufficient.}
    \subsection{Need to pay attention to the cube.}
    \subsection{Including 1/k terms helps massively.}
    \subsection{The log basis performs excellently, without being tied to a specific $n_s$.}
