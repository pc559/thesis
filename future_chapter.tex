\chapter{Future Work/Conclusions}\label{chapter:conclusion}
\section{Further SF constraints}
    \subsection{(Depending on what gets done.)}
    Words
\section{Multi-field}
    \subsection{Specific models, parameter goals.}
    Words
    \subsection{Notes on implementation.}
    Words
\section{Factor basis to assess limits}
    \subsection{(Unless this is already done)}
    Words
    \subsection{Map shape space?}
    Words
\section{User guide for Primodal code, public release of Primodal}
    \subsection{Not relevant to thesis, but could add some short notes on plans.}
    Words

\section{Discussion}
In this work we have extended the modal methods of~\cite{FergShell_1,FergShell_2,FergShell_3}
to recast the calculation of the tree-level primordial bispectrum~\eqref{in-in}
into a form that explicitly preserves its separability.
We emphasise again that this work has two main advantages over previous
numerical methods. The more immediate is that by calculating the primordial
bispectrum in terms of an expansion in some basis, the full bispectrum can
be obtained much more efficiently than through repetitive integration
separately for each $k$-configuration. The second, more important, advantage
is the link to observations.
Unlike previous numerical and semi-analytic methods,
once the shape function is expressed in some basis
as in~\eqref{shapefn},
the integral~\eqref{eq:reduced_cmb} and other computationally intensive steps involved
in estimating a particular bispectrum in the CMB, can be precomputed. Since this
large cost is only paid once per basis, once a basis
which converges well for a broad range of models
has been found, an extremely broad exploration of primordial bispectra becomes immediately feasible in the CMB.
Making explicit the $k$-dependence in this way also opens the door to vast increases in efficiency in
connecting to other observables, by precomputation using the basis set, then performing a (relatively) cheap
scan over inflation parameters.


Our work here goes beyond that of~\cite{Funakoshi} in that our careful methodology
allows us to accurately and efficiently go to much higher orders,
in particular our methods of starting the time integrals~\eqref{inin_kindep}
and of including the spatial derivative terms in the calculation.
This allowed us to present this method for feature bispectra for the first time,
demonstrating the efficient exploration of much more general primordial bispectrum phenomenology.
We have also identified and addressed the effects
of the non-physical $k$-configurations on convergence within the three-dimensional tetrapyd.
We explored, for the first time, possible basis set choices in the context
of those effects.
We showed rapid convergence on a broad range of scenarios,
including cases with oscillatory features with non-trivial shape dependence,
using our augmented Legendre polynomial basis, $\Lnsboth$.

The immediate application of this work is the efficient exploration of
bispectrum phenomenology, as our methods can much more quickly
converge to the full shape information than previous numerical methods,
which relied on calculating the shape function point-by-point, for each $k$-configuration separately.
We have implemented these methods for single field scenarios
with a varying sound speed, scenarios which
have a rich feature phenomenology. An important goal will be extending
these methods to the case of multiple-field inflation.

The next immediate application will be to directly constrain parameters of
inflationary scenarios through modal bispectrum results from the \textit{Planck} satellite~\cite{Planck_NG_2018}.
The details of the work required to directly connect our coefficients to the observed data,
and the large but once-per-basis cost of this calculation, will be detailed in
a forthcoming paper~\cite{Sohn_2021}.
CMB and LSS data from forthcoming surveys will be able to use
these separable primordial bispectra to even more precisely constrain the
parameters of inflationary scenarios.

\section{PhD project (1/2 page)}
The standard cosmological history, the $\Lambda CDM$ model, explains the structure we see in the distribution of matter (large-scale structure, LSS) and radiation (the cosmic microwave background, \textit{CMB}) in our universe. However, this model still requires initial conditions. These initial conditions – adiabatic perturbations on a uniform background, close to Gaussian, with a power spectrum that is nearly (but not exactly) scale-invariant – are provided by an epoch of “inflation”, an epoch before the Big Bang of the $\Lambda CDM$ model. 

During inflation, the universe is supposed to have expanded rapidly, stretching quantum fluctuations from small scales to large. This seeded the formation of structure and encoded correlations that we would eventually measure today.
Much information on fundamental physics has been gleaned from the two-point correlators of observations. For higher order correlators (which probe the non-linearities of interactions in the early universe) calculating the predictions of a model of inflation and comparing them to what we see in the sky is much more computationally intensive. Previous work, including that done by the Planck collaboration~\cite{Planck_NG_2018}, has managed to constrain some models of inflation through their three-point correlations (or rather through their proxy, the bispectrum).
The computational complexity was dealt with by making layers of approximations and the space
of models which could be constrained was limited.
One notable simplifying assumption is that of “separability”, which makes the process of \textit{CMB} estimation much more tractable~\cite{Komatsu_2005}.


The formalism used to calculate inflationary non-Gaussianities within an inflation scenario is
known as the in-in formalism~\cite{weinberg_in_in}.
In my PhD, I have exploited the inherent separability of the tree level in-in formalism,
using expansions in separable basis functions. Following the modal philosophy of~\cite{FergShell_3}, and building on the basic idea of~\cite{Funakoshi}, this method obviates the requirement of considering an approximate separable template,
which was a limitation of previous analyses. Instead,
the goal is to work with the full numerically calculated tree-level inflationary bispectrum of the theory
being considered---the output is not a grid of points however, but a set of coefficients of
an explicitly separable basis expansion.
This preserves the separability and therefore allows us to skip the usual template-approximation step.

I developed this separable approach into a practical and efficient numerical methodology which can be applied to a
wider and more complicated range of bispectrum phenomenology, making an important step forward towards observational
pipelines which can directly confront specific models of inflation using the full bispectrum information in a template-free analysis.

The latter part of this bispectrum estimation pipeline (\textit{CMB-BEst}~\cite{Sohn_2021}) was recently completed
by my collaborators (Wuhyun Sohn, Dr James Fergusson).
This code constrains the inflationary output of my ``\textit{Primodal}'' code using \textit{CMB} data.

In my recent paper (with E. P. S. Shellard)~\cite{probing_precision}, which has been submitted to JCAP,
we explored and described various basis choices and their advantages, and thoroughly validated the methods on single-field inflation models with non-trivial phenomenology, showing that our calculation of these coefficients is fast and accurate to high orders.


More recently I have reached the key exploitation phase of my research.
My collaborators and I have tested the integration of the \textit{Primodal} and \textit{CMB-BEst}
parts of the pipeline, and at present, my focus is on running \textit{Primodal} on the Cambridge HPC facility,
scanning models of inflation with the goal of obtaining some improved constraints on parameters of fundamental
physics using our pipeline.
Specifically, we are looking at models of DBI inflation~\cite{dbi_in_the_sky} with sharp or
oscillatory features, which are reviewed in, for example,~\cite{astro2020_features}.


\section{Research proposal (1-2 pages)}
\subsection{General Aims}
My research goals are to develop methods which will allow us to connect the fundamental physics of the early universe to observations.
My most immediate goal is to connect single-field models of inflation with the full three-point correlations of the \textit{CMB}.
Beyond that, my goal is to generalise these methods to theories with multiple fields
(which are well-motivated by ``string swampland'' considerations~\cite{achucarro_multifield1}),
and later to connect the same theories to other observables coming from large-scale structure.

My PhD has focused on the first of these goals,
calculating the bispectrum of the inflationary curvature perturbations, in a way tailored to be linked to observations.
I have developed methods to do this in the case of single field inflation.
A prime goal I would pursue in this RA position would be to extend these methods (and their implementation in code)
to the case of multiple fields. This would open the path to placing constraints on this
more general class of theories.

Readying \textit{Primodal} for a public release is another immediate goal.
This would benefit the community by providing it with a powerful tool
that can capture the full shape information of a wide class of models far more efficiently
that previous similar codes, as described in~\cite{probing_precision}.


These goals support those laid out in the recent community white paper~\cite{astro2020_png},
by improving predictions of the phenomenology of non-linearities in the very early universe.
The ability to efficiently compare the phenomenology of these scenarios to observation and thus constrain the space of theories is a very worthwhile goal, though the phenomenology is also of interest in its own right.

I will do this work in collaboration with Wuhyun Sohn, Dr James Fergusson and Prof.~E.P.S.Shellard.

\subsection{Specific Objectives}
\subsubsection{Non-linearities from inflaton self-interactions}
If the fundamental physical model of inflation is of a single field,
then the non-linearities of the self-interactions of that field may have
produced observably large non-Gaussian signatures~\cite{Tolley_2010, achucarro_eft}.
Detecting these signatures in the CMB bispectrum requires an understanding
of their form~\cite{Komatsu_2005}, which motivates the study of these dynamics.
Models of inflation that can produce observably large non-Gaussian signatures,
and are therefore constrained by the current experimental bounds~\cite{Planck_NG_2018},
include string theory-inspired models such as DBI inflation~\cite{dbi_silverstein}
and axion-monodromy~\cite{axion_monodr_review_09, Flauger_2014}
(which gives large non-Gaussianity through the resonance mechanism).
Also within reach with the current implementation of \textit{Primodal}
is the exploration of the effects of sudden
features in the inflationary dynamics, and of more general
time-dependence within the formalism of the effective field theory (EFT) of inflation. 

\subsubsection{Non-linearities from multi-field inflation}
Another source of possibly observable non-Gaussian signatures can be
found in inflation scenarios with more than one active field.
In these models, which are motivated by string theory considerations~\cite{achucarro_multifield1},
non-linearities of interactions between fields during inflation
can generate observably large non-Gaussian signals.
This is in part because they can evade the squeezed limit consistency condition~\cite{sqz_consistency}.
The extra degrees of freedom in these scenarios can generate bispectrum shapes not usually
seen in single-field models (for example, see recent work in~\cite{RP_2, Fumagalli_2019}),
making the exploration of the case of multiple fields a prime target.

\subsubsection{Optimising the separable decomposition of primordial bispectra}
The convergence of our separable expansion is a vital issue in determining the feasibility of our methods.
This convergence depends centrally on the basis choice, as we explored in~\cite{probing_precision}.
While in that work we described basis sets that could efficiently capture both the
basic shapes and some more complicated oscillatory shapes,
the question remains as to how to find the optimal separable description of a given bispectrum.

This is complementary to the philosophy of the main pipeline,
which is to find a large general basis which covers a wide range of models.
Instead, it may be possible to reduce the separable description of any given
bispectrum to a sufficiently compact form that template-free estimation
tailored to a scenario could be computationally tractable.

Additionally, the most relevant metric for convergence is the convergence of
the observables, not the bispectrum at the end of inflation---exploring the
relative efficiencies of different basis sets in this context could be very
fruitful in expanding the range of models which could be constrained.

\subsubsection{Public release of \textit{Primodal}}
The public release of \textit{Primodal} is a clear and worthwhile
deliverable of this project. While other publicly released software
for the calculation of inflationary non-Gaussianities does exist,
they are based on point-by-point methods, and are thus inefficient
in their calculation and description of the smooth bispectrum.
\textit{Primodal}, which calculates the same quantity in an intrinsically
smooth way, and whose result is tailored to comparison with observations,
would thus fill an important gap.

\subsection{Methodology}
\subsubsection{Non-linearities from inflaton self-interactions}
Using the basis sets and the separable formulation of the in-in calculation developed in my PhD,
We will reproduce and improve upon the established constraints on single-field inflation models.
An initial target will be exploring how the scale-dependence of the numerically calculated DBI bispectrum
affects the constraint on the sound speed, which was found in~\cite{Planck_NG_2018}
using a scale-invariant approximate template. We will also explore oscillatory models,
obtaining constraints using the full shape dependence, complementing the
searches done in~\cite{Planck_NG_2018} which again used approximate templates
(which exclude, for example, drifting frequencies).

Beyond this,
we will expand the focus beyond DBI inflation, resonance features and sharp features
and fully exploit the current implementation of my methods and code
by comprehensively reviewing the single-field inflation models popular in the literature.
We will explore the effects of including the full shape information in a template-free analysis,
improving bispectrum constraints on parameters of single-field inflation models.
Categorising models into distinguishability classes would be a useful initial goal.

\subsubsection{Non-linearities from multi-field inflation}
While some aspects of multi-field scenarios have been previously explored
numerically~\cite{Fumagalli_2019} using the transport method~\cite{transport_pytransport_2}
we can go beyond these explorations in two ways. Firstly, the transport method is a
point-by-point method of calculating the bispectrum, and is thus significantly
less efficient than our modal expansion method, and less suited to parameter scans.
Secondly, the resulting grid of points is not easily comparable to observations,
which is the main motivator of the separability of our method.

I will extend the methods I have already developed to the case of multiple active fields during inflation, and implement them in the \textit{Primodal} code.
I will validate my established basis sets on standard templates for multi-field bispectra,
and test their ability to capture the expected scale-dependence of the shapes,
a property usually excluded in standard templates.
Once these validations are passed, I will recast the multi-field in-in calculation
into a separable form (as I have already done for the single-field case)
and implement that calculation in code. Using that implementation,
I will explore the phenomenology of multi-field inflation,
understanding which theories and scenarios can be constrained through the bispectrum, and obtaining those constraints.

\subsubsection{Optimising the separable decomposition of primordial bispectra}
To pursue efficient separable decomposition, I intend to research techniques which have not
previously been exploited in the setting of bispectrum calculation and estimation.
One possible avenue is tensor rank decomposition techniques and other methods of tensor approximation,
which are of use in, for example, signal processing.

By using these techniques to approximate the coefficient matrix (which is the output of \textit{Primodal})
we can determine a linear combination of our basis functions which can more efficiently
describe that particular bispectrum.
I will apply these techniques to explore the general limits of separable approximations to primordial bispectra,
with a focus on convergence at the level of observations. This will make clear the widest possible set
of models that can be constrained through the bispectrum.

\subsubsection{Public release of \textit{Primodal}}
My intention is to follow best practices when preparing \textit{Primodal} for
public release. I will document the code, prepare example test cases for validation by users
on their local machines,
and release the code within a container to package the code along with its
dependencies, libraries, and system settings to ensure reproducability.

\subsection{Work plan (12 months)}
\begin{itemize}
	\item Oct-Dec
	\begin{itemize}
        \item Obtain single-field constraints using the already-validated codes \textit{Primodal} and \textit{CMB-BEst}.
        \item Exploit tensor rank decomposition for basis choice optimisation, and explore tools from other fields, e.g.~signal processing.
        \item Explore further applications to single-field models, for example time dependence within the EFT of inflation.
        \item Prepare \textit{Primodal} for public release in its single-field form.
	\end{itemize}
	\item Jan-Mar
	\begin{itemize}
        \item Test the basis sets that were used in the single-field case on multi-field templates.
            If needed, explore new basis sets to cover multi-field shapes.
        \item Generalise the methods for separable in-in calculation to the multi-field case.
        \item Prepare single-field constraint results for publication.
	\end{itemize}
	\item Apr-Jun
	\begin{itemize}
        \item Implement multi-field methods in code, extending \textit{Primodal}.
        \item Investigate multi-field phenomenology, with the goal of understanding which scenarios predict observable non-Gaussianity, and thus can be constrained.
        \item Explore the application of these methods in other contexts, such as preparing non-Gaussian initial conditions for simulations of large scale structure evolution.
	\end{itemize}
	\item Jul-Sep
	\begin{itemize}
        \item Prepare the multi-field extension of \textit{Primodal} for release.
        \item Prepare multi-field results for publication.
	\end{itemize}
\end{itemize}

\section{Career goals, benefits}
Given the methods and expertise I have developed so far in my PhD,
and the expertise in CMB bispectrum estimation in Cambridge, I am now ideally
positioned to achieve my goal of obtaining new constraints,
benefiting the department by producing leading results in an area with genuine discovery potential.

The award would benefit me by giving me the opportunity to pursue this project further
into the exploitation phase of the single-field implementation,
and the opportunity to expand the methods I have developed to the broader class
of multi-field models of fundamental early-universe physics.
Meanwhile, I would have the opportunity to further my computational
and numerical skills, and expand my skills by learning new
numerical techniques (such as tensor rank decomposition).
Having the opportunity to pursue these goals and learn these skills will greatly benefit
me in my preparation for further applications.
