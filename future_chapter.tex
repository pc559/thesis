\chapter{Conclusions}\label{chapter:conclusion}

\section{Summary}
Through the work detailed in this thesis the inherent separability of the tree level in-in formalism
has been exploited, using expansions in separable basis functions.
Following the modal philosophy of~\cite{FergShell_1,FergShell_2,FergShell_3}, and building on the basic idea of~\cite{Funakoshi},
this method obviates the requirement of considering an approximate separable template,
which was a limitation of previous analyses. Instead,
the goal is to work with the full numerically calculated tree-level inflationary bispectrum of the scenario
being considered. The output is not a grid of points, but a set of coefficients of
the explicitly separable basis expansion.
This preserves the separability and therefore allows us to skip the usual template-approximation step.
We have:
\begin{itemize}
    \item developed practical sets of such basis functions, showing the overall method
to be feasible.
    \item developed efficient and robust methods of applying the in-in formalism
to an inflation scenario, to generate shape coefficients for some given basis.
    \item applied these methods to a particular
inflation scenario, and used the separability of our result to connect to the methods of~\cite{Sohn_2021}.
This allowed us to place a constraint on that scenario, using $\planck$ $\cmb$ data.
\end{itemize}
Through this work we developed this separable approach into a practical and efficient numerical methodology,
allowing it to be applied to a
wider and more complicated range of bispectrum phenomenology than previously.
This is an important step forward towards observational
pipelines which can directly confront specific models of inflation using the full bispectrum
information in a template-free analysis.
These methods and basis sets we developed have been implemented in the $\primodal$ code, which calculates
the shape coefficients for a given inflation scenario, in a given basis.
We have explored and contrasted the advantages of the various basis sets we described, and thoroughly
validated our methods on single-field inflation models with non-trivial phenomenology.
This validation showed that
our calculation of these coefficients is fast and accurate to high orders.


The latter part of this bispectrum estimation pipeline
connects the shape coefficient output of $\primodal$ to the $\cmb$.
This part of the pipeline is described and
validated in~\cite{Sohn_2021} and implemented in the $\cmbbest$ code.
In chapter~\ref{chapter:constraints} we tested the integration of the $\primodal$ and $\cmbbest$
parts of the pipeline.
We scanned across a fundamental parameter, $\bir$, of the $\dbi$ model of inflation.
Our goal was to obtain a constraint on this parameter using our pipeline---we found 
$\bir\le0.46$ at $95\%$ confidence using $\planck$ temperature data.
This differed from the equivalent constraint found by the $\modal$ estimator
in the $\planck$ analysis, but the results were sufficiently in agreement that we
can take this as a broad validation of our pipeline.


\section{Discussion}
In this work we have extended the modal methods of~\cite{FergShell_1,FergShell_2,FergShell_3}
to recast the calculation of the tree-level primordial bispectrum~\eqref{inin_integral}
into a form that explicitly preserves its separability.
We emphasise again that this work has two main advantages over previous
numerical methods. The more immediate is that by calculating the primordial
bispectrum in terms of an expansion in some basis, the full bispectrum can
be obtained much more efficiently than through repetitive integration
separately for each $k$-configuration. The second (and more important)
advantage is the link to observations.
Unlike previous numerical and semi-analytic methods,
once the shape function is expressed in some basis as in~\eqref{goal},
the integral~\eqref{eq:reduced_cmb} and other computationally intensive steps involved
in estimating a particular bispectrum in the $\cmb$, can be precomputed. Since this
large cost is only paid once per basis, once a basis
which converges well for a broad range of models
has been found, an extremely broad exploration of primordial bispectra becomes immediately feasible in the $\cmb$.
%Making explicit the $k$-dependence in this way also opens the door to vast increases in efficiency in
%connecting to other observables, by precomputation using the basis set, then performing a (relatively) cheap
%scan over inflation parameters.


Our work here goes beyond that of~\cite{Funakoshi} in that our careful methodology
allows us to accurately and efficiently go to much higher orders.
%in particular our methods of starting the time integrals~\eqref{inin_kindep}
%and of including the spatial derivative terms in the calculation.
This allowed us to present this method for feature bispectra for the first time,
with linear oscillations, logarithmic oscillations, and complicated shape dependence.
%demonstrating the efficient exploration of much more general primordial bispectrum phenomenology.
We also identified and addressed the effects
of the non-physical $k$-configurations on convergence within the three-dimensional tetrapyd.
We explored, for the first time, possible basis set choices in the context of those effects.
We showed rapid convergence on a broad range of scenarios,
including cases with oscillatory features with non-trivial shape dependence,
using our augmented basis sets.


The immediate application of this work is the efficient exploration of
bispectrum phenomenology, as our methods can much more quickly
converge to the full shape information than previous numerical methods,
which relied on calculating the shape function point-by-point, for each $k$-configuration separately.
We have implemented these methods for single field scenarios
with a varying sound speed, scenarios which
have a rich feature phenomenology. An important goal will be extending
these methods to the case of multiple-field inflation.
These goals support those laid out in the recent community white paper~\cite{astro2020_png},
by improving predictions of the phenomenology of non-linearities in the very early universe.
The ability to efficiently compare the phenomenology of these scenarios to observation and thus
constrain the space of theories is a very worthwhile goal, though the phenomenology is also of interest in its own right.


The advantage of our methods is that it will enable the use of $\planck$ data
(and data from future experiments)
to directly constrain parameters of inflationary scenarios which do not have
standard templates, thus obtaining new constraints on inflationary parameters.


Using the basis sets and the separable formulation of the in-in calculation developed in this work,
it will be possible to reproduce and improve upon the established constraints on single-field inflation models
and oscillatory models,
obtaining constraints using the full shape dependence, complementing the
searches done in~\cite{Planck_NG_2018} which again used approximate templates
(which exclude, for example, drifting frequencies).



