%\newcommand{\eps}{\epsilon}
\newcommand{\Pmax}{p_\text{max}} 	
\newcommand{\Hint}{H_{int}}
\newcommand{\kbar}{\bar{k}}
\newcommand{\Delk}{\Delta}
\newcommand{\Sumk}{\Sigma}
\newcommand{\Qrot}{Q_{pq}(\tau_s)}
\newcommand{\shapecor}{\mathcal{S}}
\newcommand{\ampcor}{\mathcal{A}}
\newcommand{\totalcor}{\mathcal{E}}
\newcommand{\threeLs}{L_p(\kbar_1)L_q(\kbar_2)L_r(\kbar_3)}
\newcommand{\threePs}{P_p(\kbar_1)P_q(\kbar_2)P_r(\kbar_3)}
\newcommand{\threeQs}{Q_p(k_1)Q_q(k_2)Q_r(k_3)}
\newcommand{\Lbasic}{\mathcal{P}_0}
\newcommand{\Linvk}{\mathcal{P}_1}
\newcommand{\Lnsinv}{\mathcal{P}^{n_s}_1}
\newcommand{\Lnsboth}{\mathcal{P}^{n_s}_{01}}
\newcommand{\Linvksq}{\mathcal{P}_2}
\newcommand{\Lns}{\mathcal{P}^{n_s}_2}
\newcommand{\Fbasic}{\mathcal{F}_0}
\newcommand{\Finvk}{\mathcal{F}_1}
\newcommand{\Finvksq}{\mathcal{F}_2}
\newcommand{\quadpot}{V_{\phi^2}(\phi)}
%\newcommand{\threeqs}{q_p(\kbar_1)\,q_r(\kbar_2)\,q_s(\kbar_3)}
\newcommand{\threeqs}{q_p(k_1)\,q_r(k_2)\,q_s(k_3)}
\newcommand{\threeqstilde}{q_{\tilde{p}}(k_1)\,q_{\tilde{r}}(k_2)\,q_{\tilde{s}}(k_3)}
\newcommand{\kmin}{{k_\text{min}}}
\newcommand{\kmax}{{k_\text{max}}}
\newcommand{\fnl}{f_{NL}}
\newcommand{\fnllocal}{f^{local}_{NL}}
\newcommand{\fnlequil}{f^{equil}_{NL}}
\newcommand{\fnlortho}{f^{ortho}_{NL}}

\chapter{Constraints}\label{chapter:constraints}
\section{Connecting to $\cmbbest$}
\textcolor{red}{We don't get an exact match, maybe due
to $140$ Gaussian simulations vs. Modal's $~300$.
Or perhaps it is simply a statistical fluctuation.}
    The goal here is to validate our pipeline by reproducing the $\planck$~result
    from~\cite{Planck_NG_2018}, equation (55):
    \begin{align}
        c_s^{DBI}&\ge0.079\qquad(95\%,~T~\text{only}).
    \end{align}
    We would not expect to reproduce this exactly due to
    \textcolor{red}{Lack of maps? Different approximations? Numerics?}
    Though we can estimate a reasonable disagreement by looking at
    the expected scatter between the $\planck$ estimators, and by calculating
    \textcolor{red}{$\sigma$??}.

\section{Set-up of the scan}
    We use equation~\eqref{eq:dbi_warp}, with $\beta_{IR}\in[0.1885, 0.58]$.
    We find that $\beta_{IR}=0.331$ produces $c_s^{*}=0.0794$,
    which is close to the above constraint.
    Thus, we roughly expect to find that $\beta_{IR}<0.331$ is ruled out
    (for all the other scenario parameters held fixed).
    We also have that $\ln\left(10^{10}A_s\right)$ for each scenario
    is within $3.044\pm0.014$ across the scan,
    and that $n_s^{*}$ is within $0.9649\pm0.0042$ across the scan.
    This set-up produces a scenario with ($*$ denotes horizon crossing of the pivot scale):
%    \begin{tabular}{lrrrrrrrrrr}
%        \toprule
%        $\beta_{IR}$ &    $c_s^{*}$ &  $\varepsilon_s^{*}$ &   $\varepsilon^{*}$ &   $n_s^{*}$ &  $n_{NG}^{*}$ &   $\phi^{*}$ &     $H^{*}$ &   $\eta^{*}$ \\
%        \midrule
%        $1.89\times 10^{-1}$  &  $1.39\times 10^{-1}$  &  $  8.57\times 10^{-3}$  &  $7.44\times 10^{-5}$  &  $-3.50\times 10^{-2}$  &  $-8.72\times 10^{-2}$  &  $5.19\times 10^{-1}$  &  $1.31\times 10^{-6}$  &  $2.63\times 10^{-2}$ \\
%        $5.80\times 10^{-1}$  &  $4.50\times 10^{-2}$  &  $  8.67\times 10^{-3}$  &  $2.31\times 10^{-4}$  &  $-3.63\times 10^{-2}$  &  $-8.99\times 10^{-2}$  &  $5.15\times 10^{-1}$  &  $1.30\times 10^{-6}$  &  $2.71\times 10^{-2}$ \\
%        \bottomrule
%    \end{tabular}
    \\
    \\
\begin{table}[h!]
  \begin{center}
    \begin{tabular}{lrrrrrrr}
        \toprule
        $\beta_{IR}$ &    $c_s^{*}$ &  $\varepsilon_s^{*}$ &   $\varepsilon^{*}$ &   $n_s^{*}$ &  $n_{NG}^{*}$\\
        \midrule
        $1.89\times 10^{-1}$  &  $1.39\times 10^{-1}$  &  $  8.57\times 10^{-3}$  &  $7.44\times 10^{-5}$  &  $9.650\times 10^{-1}$  &  $-8.72\times 10^{-2}$\\
        $5.80\times 10^{-1}$  &  $4.50\times 10^{-2}$  &  $  8.67\times 10^{-3}$  &  $2.31\times 10^{-4}$  &  $9.637\times 10^{-1}$  &  $-8.99\times 10^{-2}$\\
        \bottomrule
    \end{tabular}
    \caption{Summary of scenario parameters across the scan.}\label{tab:scan_summary}
  \end{center}
\end{table}
    \\
    \\
\begin{table}[h!]
  \begin{center}
    \begin{tabular}{lrrr}
        \toprule
        $\beta_{IR}$ &  $\phi^{*}$ &     $H^{*}$ &   $\eta^{*}$ \\
        \midrule
        $1.89\times 10^{-1}$  &  $5.19\times 10^{-1}$  &  $1.31\times 10^{-6}$  &  $2.63\times 10^{-2}$ \\
        $5.80\times 10^{-1}$  &  $5.15\times 10^{-1}$  &  $1.30\times 10^{-6}$  &  $2.71\times 10^{-2}$ \\
        \bottomrule
    \end{tabular}
    \caption{Summary of scenario parameters across the scan.}\label{tab:scan_summary2}
  \end{center}
\end{table}
    \\
    \\
    We now show convergence results for $\Lnsinv$ with $\Pmax=30$.
    For this basis the match to the template is good in the equilateral limit, but quite poor in the squeezed limit.
    \\
    \\
\begin{table}[h!]
  \begin{center}
    \begin{tabular}{lrrrr}
        \toprule
        $\beta_{IR}$ & Sum Template & Product Template & Bare Template & With $\Pmax=25$ \\
        \midrule
        $1.89\times 10^{-1}$  &  $6.15\times 10^{-3}$  &  $7.22\times 10^{-3}$  &  $5.16\times 10^{-2}$  &  $5.3\times 10^{-3}$ \\
        $5.80\times 10^{-1}$  &  $4.51\times 10^{-3}$  &  $3.63\times 10^{-3}$  &  $5.28\times 10^{-2}$  &  $2.6\times 10^{-3}$ \\
        \bottomrule
    \end{tabular}
    \caption{
        Comparing the $\Lnsinv(\Pmax=30)$ result compared to
        templates~\eqref{dbi_shape},~\eqref{dbi_sum_shape} and~\eqref{dbi_prod_shape}.
        We also refit the result with $\Lnsinv(\Pmax=25)$, and compare with the full
        result to estimate the convergence at the primordial level.
    }\label{fig:template_errors}
  \end{center}
\end{table}
    \\
    \\
    The convergence in the $\scalingbasis$ basis is better,
    falling in the range $[1.02\times 10^{-4}, 9.24\times 10^{-4}]$.
    However, for this analysis the $\cmbbest$ code had only been run for
    the $\Lnsinv$ basis. We see that it is sufficient in any case.
    When we examine the convergence to the sum~\eqref{dbi_sum_shape}
    and product~\eqref{dbi_prod_shape} scaling templates,
    in figure~\ref{fig:dbi_primodal_scan_template_corrs_log30},
    we see that neither is obviously the better match to the numerical result.
    This is due to the numerical result having a non-zero squeezed limit
    coming from the usual slow-roll suppressed local-type contributions
    (as in~\eqref{malda_shape}) which are neglected in the DBI templates.
    \\
\begin{figure}[!pth]
\centering
\includegraphics[width=\columnwidth]{dbi_scan_template_corrs_plots/dbi_primodal_scan_template_corrs_log30.png}
\caption{
    The $\scalingbasis$ basis converges well across the scan range.
    We see that the bare DBI template is a poor match to the true numerical result.
    This is mostly due to the error in the overall magnitude.
    Once this is corrected, we see that the numerical result matches the
    approximate template to better than $1\%$. As the convergence of the
    numerical result is better than $0.1\%$ for the $\scalingbasis$ basis
    we can see that sum scaling~\eqref{dbi_sum_shape} and the
    product scaling~\eqref{dbi_prod_shape} perform
    comparably in matching the numerical result. This is mostly
    due to those templates neglecting the usual slow-roll suppressed
    contributions (as in~\eqref{malda_shape}),
    which do in fact become relevant to the primordial
    bispectrum deep enough into the squeezed limit, due to their local-type shape.
}\label{fig:dbi_primodal_scan_template_corrs_log30}
\end{figure}
\begin{figure}[!pth]
\centering
\includegraphics[width=\columnwidth]{dbi_scan_template_corrs_plots/dbi_primodal_scan_template_corrs_p1ns30}
\caption{
    The $\Lnsinv$ basis is sufficiently convergent across the scan range
    to obtain the desired constraint.
    We see that the convergence error is only slightly better than the error
    in the slow-roll corrected templates.
}\label{fig:dbi_primodal_scan_template_corrs_p1ns}
\end{figure}
    \\
\section{Compare convergence at primordial level to convergence at $f_{NL}$ level}
    We will compare primordial convergence to CMB convergence,
    by comparing the $\Pmax=29$ and $\Pmax=30$ results.
    This will show that the CMB convergence is slightly better, i.e.\ that
    the squeezed limit (which is where the primordial shape converges most slowly)
    is suppressed.
    At the $\cmb$ level, the convergence is $O(10^{-5})$. What is it for $\Pmax=29$?
    at the primordial level?
    For DBI, and also for DBI resonance?
\section{Validate this convergence by reproducing $\planck$ constraint using DBI template decomposition.}
    \begin{figure}[htbp!]
        \centering
        \includegraphics[width=0.9\textwidth]{wuhyun_plots/dbi_sound_speed_scan_annotated.pdf}
        \caption{
            Our constraint on DBI inflation. Here $c_s^{DBI}$ is not an input parameter
            (unlike in the template case), instead it is time dependent, and the plotted
            value is taken from the horizon crossing of the pivot scale. We find that $\beta_{IR}<0.5$
            %0.464
            is outside of our $2\sigma$ confidence interval. This plot was obtained by
            scanning across values of $\beta_{IR}$ and calculating the corresponding primordial bispectra
            using $\primodal$, then projecting those bispectra onto the $\cmb$
            and comparing them to the $\planck$ $\cmb$ temperature data using
            $\cmbbest$. Since the amplitude is fixed by the scenario, we rule out a
            scenario by ruling out $f_{NL}^{DBI}=1$.
            Note that the constraint obtained for $\Pmax=30$ and $\Pmax=29$ is identical,
            so we can be confident that our constraint has converged in $\Pmax$.
            \textcolor{red}{Could add vertical line at $\planck$ constraint.
            How sure are we that the factor of two is a coincidence??
            Legendre probably needs to be replaced with $\Lnsinv$.
            I'd probably prefer $\Pmax=25$ comparison, as sometimes step convergence is
            a thing (and also that would demonstrate robustness of including high order modes).
            }
        }\label{fig:dbi_sound_speed_scan}
    \end{figure}
\section{Then, check how the scaling from the real numerical result affects this.}
    Here we will compare template decompositions to Primodal results.
    This will tell us how large the slow-roll corrections are to the
    final CMB result, but not anything about the primordial convergence.
    The main slow-roll corrections are a correction to the amplitude,
    and a deviation from perfect scale-dependence.
%\section{EFT stuff, build off Enrico's recent work.}
%    Words
