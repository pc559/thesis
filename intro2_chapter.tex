%\newcommand{\eps}{\epsilon}
\newcommand{\Pmax}{p_\text{max}} 	
\newcommand{\Hint}{H_{int}}
\newcommand{\kbar}{\bar{k}}
\newcommand{\Delk}{\Delta}
\newcommand{\Sumk}{\Sigma}
\newcommand{\Qrot}{Q_{pq}(\tau_s)}
\newcommand{\shapecor}{\mathcal{S}}
\newcommand{\ampcor}{\mathcal{A}}
\newcommand{\totalcor}{\mathcal{E}}
\newcommand{\threeLs}{L_p(\kbar_1)L_q(\kbar_2)L_r(\kbar_3)}
\newcommand{\threePs}{P_p(\kbar_1)P_q(\kbar_2)P_r(\kbar_3)}
\newcommand{\threeQs}{Q_p(k_1)Q_q(k_2)Q_r(k_3)}
\newcommand{\Lbasic}{\mathcal{P}_0}
\newcommand{\Linvk}{\mathcal{P}_1}
\newcommand{\Lnsinv}{\mathcal{P}^{n_s}_1}
\newcommand{\Lnsboth}{\mathcal{P}^{n_s}_{01}}
\newcommand{\Linvksq}{\mathcal{P}_2}
\newcommand{\Lns}{\mathcal{P}^{n_s}_2}
\newcommand{\Fbasic}{\mathcal{F}_0}
\newcommand{\Finvk}{\mathcal{F}_1}
\newcommand{\Finvksq}{\mathcal{F}_2}
\newcommand{\quadpot}{V_{\phi^2}(\phi)}
%\newcommand{\threeqs}{q_p(\kbar_1)\,q_r(\kbar_2)\,q_s(\kbar_3)}
\newcommand{\threeqs}{q_p(k_1)\,q_r(k_2)\,q_s(k_3)}
\newcommand{\threeqstilde}{q_{\tilde{p}}(k_1)\,q_{\tilde{r}}(k_2)\,q_{\tilde{s}}(k_3)}
\newcommand{\kmin}{{k_\text{min}}}
\newcommand{\kmax}{{k_\text{max}}}
\newcommand{\fnl}{f_{NL}}
\newcommand{\fnllocal}{f^{local}_{NL}}
\newcommand{\fnlequil}{f^{equil}_{NL}}
\newcommand{\fnlortho}{f^{ortho}_{NL}}

\chapter{Probing inflation}\label{chapter:intro_bispectra}
The primordial bispectrum is one of the main
characteristics used to distinguish between models of inflation. While it is well
known that the physics of inflation must have been extremely close
to linear, and the initial seeds of structure it laid down
very close to Gaussian, there is expected to have been some level of coupling
between the Fourier modes of the perturbations.
In the simplest example of an inflation model this is
expected to be unobservable~\cite{Maldacena},
but the possibility remains that inflation was driven by
more complex physics that may have left an observable imprint on our universe today.
Some models of inflation have interactions that predict non-Gaussian
correlations at observable levels. Ways this can happen include
self-interactions~\cite{px_burrage,dbi_in_the_sky},
interactions between multiple fields~\cite{Byrnes_2010},
sharp features~\cite{adshead}
and periodic features~\cite{flauger_pajer_resonant}.
However, constraining such imprints is extremely difficult observationally.
Even once the data has been obtained, using existing methods it is
extremely computationally intensive to translate this into constraints
on specific inflation scenarios. Much progress has been made by course-graining
the model space into a small number of approximate templates,
and leveraging the simplifying characteristic of separability
with respect to the three parameters of the bispectrum~\cite{Komatsu_2005, Munchmeyer_2014}.


The primordial bispectrum is the Fourier equivalent of the
three-point correlator of the primordial curvature perturbation.
If this field is Gaussian, the bispectrum vanishes, so
it is a valuable measure of the interactions in play during inflation.
If some inflation model predicts a bispectrum that is sufficiently well approximated by
the standard separable templates, the constraints on those standard templates
can be translated into constraints on the parameters of the model.
The fact that all primordial templates estimated thus far from the CMB
are consistent with zero has already provided such constraints
in certain scenarios~\cite{Planck_NG_2015, Planck_NG_2018}.
With this high-precision \planck~data, and data from forthcoming experiments
such as the Simons Observatory (SO)~\cite{simons}
and CMB-S4~\cite{abazajian2016cmbs4},
robust pipelines must be developed to circumvent the computational difficulties and
extract the maximum amount of information possible.
Due to the nature of bispectrum estimation in the CMB and
LSS~\cite{lss_baldauf,lss_karagiannis,chen_future_lss,Scoccimarro_2012}
constraining an arbitrary template is difficult.
Our aim in this work is to develop the inflationary part
of a pipeline to allow to efficiently test a much broader range of models.
In this work, we explore shapes arising from tree-level effects in single field models.
We do this numerically, allowing quantitative results for a broad
range of models, and avoiding extra approximations.
Our general aim is to apply the modal philosophy of~\cite{FergShell_1,FergShell_2,FergShell_3}
to calculating primordial bispectra.
This modal philosophy is a flexible method that has broadened the range of constrained
bispectrum templates, by expanding them in a carefully chosen basis.
The Modal estimator is thus capable of constraining
non-separable templates, while the KSW estimator cannot.
In this work we exploit the intrinsic separability of the
tree-level in-in formalism to apply these methods at the level of inflation.
Expressing the primordial bispectrum in a separable
basis expansion leads to vast increases in efficiency both at the primordial
and late-universe parts of the calculation.
The main advantage is that expressing the primordial shape function
in this way reduces the process of bispectrum estimation in the CMB to a
cost which is large, but need only be paid once per basis,
not per scenario.
A proof of concept of this approach at the primordial level was presented in~\cite{Funakoshi},
and the details of the bispectrum estimation part will be detailed in~\cite{Sohn_2021}.
We go beyond the work of~\cite{Funakoshi} both in developing the choice of basis
(the feasibility of the method depending vitally on the chosen basis
achieving sufficiently fast convergence in a broad range of interesting models)
and in the methods we use to allow us to go to much higher order in our modal expansion,
allowing us to apply the method to feature bispectra for the first time.



\section{Perturbations about the background}
    We can get a lot of information on the evolution of the perturbations
    from observations, and that tells us about the history of universe,
    which tells us about its content, etc\ldots
    \newpage
    The sound speed $c_s$. What is the intuition for $c_s$?
    What does a photon path look like on a $\tau_s-x$ plot?
    \newpage
    \subsection{Initial conditions for the perturbations}
    The Bunch-Davies vacuum.
    The vacuum is not unique. We choose it so that it
    asymptotes to the Minkowski vacuum in the limit $\tau\to-\infty$.
    \newpage
    \subsection{Evolution of the perturbations}
    %https://nms.kcl.ac.uk/eugene.lim/AdvCos/lecture2.pdf
    %https://www.mpi-hd.mpg.de/lin/events/group_seminar/inflation/schmidt.pdf
    Derivation of the Mukhanov-Sasaki equation, behaviour at various times.
    Second order action, in comoving gauge.
    Expanding the operator in Fourier modes.
    Quantization.
    In conformal time, we obtain
    \begin{align}\label{modefneqn_tau}
        \frac{\partial^2 u_k}{\partial \eta^2} + \left(c_s^2k^2 - \frac{1}{z}\frac{d^2 z}{d \eta^2}\right)u_k = 0
    \end{align}
    where $z^2 = 2a^2\varepsilon/c_s^2$ and $u=z\mathcal{R}$.
    We can see that at early times$\ell$.
    \newpage
    The Mukhanov-Sasaki equation in $N$, $\tau_s$---no numerics here, though, just ref forward.
    \newpage
    \subsection{Connecting the primordial power spectrum to observables}
    Define the $\cmb$ power spectrum, discuss constraints.
\newpage
\section{The primordial bispectrum}
    \subsection{The shape function}
    Shape versus scale dependence.
    Consequences of isotropy, homogeneity.
    Define the shape function.
The primordial bispectrum is usually written as:
\begin{align}
{
\left< \zeta_{\bf{k_1}}\zeta_{\bf{k_2}}\zeta_{\bf{k_3}} \right>
= (2\pi)^3\delta^{(3)}(\mathbf{k_1}+\mathbf{k_2}+\mathbf{k_3})B(k_1,k_2,k_3)
}
\end{align}
where $\zeta_{\bf{k}}$ is a Fourier mode of the standard gauge invariant curvature perturbation.
The delta function comes from demanding statistical homogeneity;
demanding statistical isotropy restricts the remaining dependence to the magnitudes of the vectors.
We denote the magnitude of $\mathbf{k_i}$ as $k_i$.
This leaves us with a function of three parameters,
$k_1,k_2,k_3$.
It is useful to define the dimensionless shape function:
\begin{align}\label{shapefn}
{
    S(k_1,k_2,k_3) = (k_1k_2k_3)^2B(k_1,k_2,k_3).
}
\end{align}
The bispectrum is only defined where the triangle condition
\begin{align}\label{triangle_condition}
    \mathbf{k_1}+\mathbf{k_2}+\mathbf{k_3} &= 0,
\end{align}
is satisfied, which implies that the triangle inequality must hold
\begin{align}\label{triangle_inequality}
    k_1+k_2 &\geq k_3~\text{and cyclic perms}.
\end{align}
The space of configurations we are interested in is therefore
reduced from the full cube $[\kmin,\kmax]^3$
to a tetrapyd (illustrated in figure~\ref{slice_explained}),
the intersection of that cube with the
tetrahedron that satisfies~\eqref{triangle_inequality}.
This has important implications that we will explore in
chapter~\ref{chapter:decomp}. 


The amplitude of a bispectrum shape is usually
quoted in terms of some $\fnl^F$ parameter.
We can schematically define $\fnl^F$ for some template $F$ as follows:
\begin{equation}\label{def:fnl}
{
B^{F}(k_1,k_2,k_3) = \fnl^{F}\times F(k_1,k_2,k_3)
}
\end{equation}
where $F$ contains the dependence on the $k$-configuration.
This definition coincides with the definitions of
$\fnllocal$, $\fnlequil$ and $\fnlortho$
when $F$ is (respectively) the local (see \eqref{local_shape}),
equilateral (see \eqref{equil_shape}) and orthogonal templates,
as defined in~\cite{Planck_NG_2015}.

    \newpage
    \subsection{The definitions of $f_{NL}$}
    Describe the various definitions of $f_{NL}$ in different
    parts of the literature.
    \newpage
    \subsection{Non-linearity during inflation}
    Connect (self-)interactions during inflation to non-Gaussianity.
    \newpage
\section{Connecting the primordial bispectrum to observables}
    Connecting to the CMB---transfer functions, etc.
\newpage
    \section{Calculating the primordial bispectrum from an inflation scenario}
    (Coming from part iii level, a derivation of the tree-level in-in formalism.)
    The standard starting point for calculating
higher-order correlators for models of inflation is the in-in formalism~\cite{Maldacena,weinberg_in_in}.
The in-in formalism takes the time evolution of the interaction picture mode
functions as an input for calculating the bispectrum.
At tree-level, the in-in formalism gives us the
following expression:
\begin{align}
{
    \left< \zeta_{\mathbf{k_1}}(\tau)\zeta_{\mathbf{k_2}}(\tau)\zeta_{\mathbf{k_3}}(\tau) \right>
=-i\int_{-\infty(1-i\varepsilon)}^{\tau}d\tau'a(\tau')
    \left<0\lvert \zeta_{\mathbf{k_1}}(\tau)\zeta_{\mathbf{k_2}}(\tau)\zeta_{\mathbf{k_3}}(\tau)\Hint(\tau') \rvert0\right>+c.c\label{in-in}
}
\end{align}
where all the operators on the right-hand side are in the interaction picture
and $\Hint$ is the interaction Hamiltonian, containing terms cubic in $\zeta$.
From this calculation we obtain the dimensionless shape function $S(k_1,k_2,k_2)$,
defined in~\eqref{shapefn},
which is then used as input into~\eqref{eq:reduced_cmb}.
As an example, if one takes $\Hint\propto\dot{\zeta}^3$, this set-up can produce the standard EFT shape
\begin{align}\label{example_eft2}
    S(k_1, k_2, k_3) = \frac{k_1k_2k_3}{(k_1+k_2+k_3)^3}.
\end{align}

The central point, as noticed in~\cite{Funakoshi}, is that the
integrand of~\eqref{in-in} is intrinsically separable
in its dependence on $k_1$, $k_2$ and $k_3$, and that the time integral
can be done in such a way as to preserve this separability.
This intrinsic separability has clearly been lost in
the example in~\eqref{example_eft2},
but can be regained (to arbitrary precision) by approximating it
with a sum of separable terms. Our general aim will be to directly calculate
this sum for a broad range of inflation models.

We now briefly outline the set-up of the standard calculation.
The Lagrangian is expanded in the perturbations and used to obtain the Hamiltonian.
The Hamiltonian is split into $H_0$ and $\Hint$.
The first part is used to evolve the interaction picture fields, $\zeta_I$,
which we will simply refer to as $\zeta$, as in~\eqref{in-in}.
The perturbations see an interaction Hamiltonian $\Hint$,
of which we will consider the part cubic in the perturbations,
with time dependent coefficients due to the evolution of the background fields.
The perturbations are assumed to be initially in the Bunch-Davies vacuum,
but the non-linear
evolution introduces correlations between the modes.
As the modes cross the horizon they begin to behave classically
and eventually freeze out.


There is some freedom in how to represent the interaction Hamiltonian,
as the equation of motion of the free fields can be used, along with integration by parts~\cite{rp_integ_by_parts}.
This can be used, as pointed out in~\cite{Funakoshi}, to avoid numerically difficult cancellations.
Some presentations of this calculation use a field redefinition to eliminate terms
proportional to the equation of motion from the Lagrangian.
As pointed out in~\cite{px_burrage},
this is unnecessary as these terms will never contribute to the bispectrum result.
In fact, in some scenarios (such as resonant models) it introduces a numerically difficult
late time cancellation between a term in the interaction Hamiltonian and the
correction to the correlator that adjusts for the field redefinition.


The bispectrum arising from a single field inflation model,
with a canonical kinetic term, slowly rolling, turns out to produce
unobservably small non-Gaussianity~\cite{Maldacena}.
However, by breaking these assumptions large signals can arise.
These signals are usually calculated using~\eqref{in-in} within tailored approximations.
The results are not always separable, so further approximations must then be made to
allow comparison with the CMB.
    \section{Calculating the interaction Hamiltonian}
    General overview of how the interaction Hamiltonian is obtained.
    \newpage
    \section{Self-interactions}
    Simpler derivation neglecting the metric perturbations to begin with.
    \newpage
    Discuss DBI, then $P(X, \phi)$.
    \newpage
    \section{The Maldacena calculation}
    Review, with metric perturbations.
    \newpage
    \section{The field-redefinition}
    Discuss field-redefinition being unnecessary, as per~\cite{px_burrage}.
    \newpage
    \section{Templates}
    Using $K_{pq}$ notation.
    Link to Enrico's symmetric polynomials, reference discussion under
    equation (57) in~\cite{Fergusson_2010}.
    \newpage
    \subsection{Basic templates, $f_{NL}$}
    The three basic phenomenological templates, \planck.
    $f_{NL}$ from the data analysis perspective,
    and from the primordial perspective.
    \section{Shapes}
    \subsection{Basic shapes}
    Maldacena, DBI.
    \subsection{$P(X, \phi)$, EFT}
    There is an extensive literature on the calculation
of bispectra from models of inflation~\cite{chen_easther_lim_1,chen_easther_lim_2,chen_ng_0605,seery_ng_0503,px_burrage,adshead,flauger_pajer_resonant,features_bartolo,bdy_passaglia}.
Multi-field models can produce large
correlations between modes of very different scales;
non-canonical kinetic terms can reduce the sound speed of the perturbations,
boosting both the smooth non-Gaussian correlations, and any
features which may be present~\cite{dbi_adshead,dbi_in_the_sky,dbi_miranda,dbi_silverstein,dbi_step_miranda,chen_folded_resonant,osc_avila};
effectively single-field models with imaginary sound speeds can generate a bispectrum
mostly orthogonal to the usual equilateral and local templates~\cite{RP_1}.
The methods outlined in this thesis have been implemented
and tested for single-field models,
with multi-field models being a prime target for future work.
We will work with an inflaton action of the form
\begin{align}
S = \int d^4x \sqrt{-g}P(X,\phi)
\end{align}
with $X=-\frac{1}{2}g^{ab}\nabla_a \phi\nabla_b \phi$.
We work with the number of e-folds, $N$, as our time variable:
$x'=\frac{dx}{dN}=a\frac{dx}{da}$.
We define the Hubble parameter and the standard ``slow-roll'' parameters:
\begin{equation}
\label{slowrollparams}
\begin{split}
    H = \frac{d\ln a}{dt}	\,,
    \qquad
    \eps &= -\frac{d\ln H}{dN}	\\
    \eta = \frac{d\ln \varepsilon}{dN}	\,,
    \qquad
    \eps_s &= +\frac{d\ln c_s}{dN}	\,.
\end{split}
\end{equation}
though we make no assumption that these are actually small.
$c_s$ is the sound speed of the theory, which can vary with time:
\begin{align}
c_s=\frac{P,_X}{P,_X+2XP,_{XX}}.
\end{align}
The background quantities are evolved according to the Friedmann equations,
which are set with consistent initial conditions.
The equation of motion for the perturbations is:
\begin{align}\label{modefneqn}
\zeta_k''+(3-\varepsilon+\eta-2\varepsilon_s)\zeta_k'+\frac{c_s^2k^2}{a^2H^2}\zeta_k=0
\end{align}
where $c_s=1$ for standard canonical inflation.
We use standard Bunch-Davies initial conditions,
which leads us to impose the following condition deep in the horizon:
\begin{align}\label{bd_ic}
\zeta_k = \frac{i}{a}\sqrt{\frac{c_s}{4\varepsilon k}} e^{-ik\tau_s}
\end{align}
where we define $\tau_s$ through $\tau_s'=\frac{c_s}{aH}$
in analogy with the usual $\tau$ with $\tau'=\frac{1}{aH}$.
The solution in slow-roll (without features) is then approximately
\begin{align}\label{modefnsapprox}
    \zeta_k \propto (1+ik\tau_s)e^{-ik\tau_s}.
\end{align}
At leading order in slow-roll the power spectrum is~\cite{mukhanov_1999,chen_ng_0605}:
\begin{align}
P^{\zeta}(k) = \frac{1}{8\pi^2}\frac{H^2}{c_s\varepsilon},
\end{align}
where the right hand side is evaluated at $c_{s}k=aH$.
The spectral index is (also to leading order):
\begin{align}
n_s-1 = -2\varepsilon-\eta-\varepsilon_s.
\end{align}
Similarly to~\cite{Funakoshi}, at early times we extract the factor of $e^{-ik\tau_s}$ from the mode functions
and numerically evolve $\zeta_ke^{ik\tau_s}$\footnote{
    In fact~\cite{Funakoshi} extracts a factor of $e^{-ikc_s(\tau)\tau}$, losing efficiency
    due to slow-roll corrections.
}.
Unless interrupted, this prefactor decays exponentially.
Eventually we switch to evolving $\zeta_k$ directly.
For featureless slow-roll inflation the timing of the switch is simple;
so long as it is around horizon crossing, or a couple of e-folds after,
the precise location will not affect the result.
This becomes trickier when we are dealing with a model with
a step feature, for example.
Here, we found that navigating the feature in the first set of variables
causes difficulty for the stepper.
Switching to $\zeta_k$ before the onset of the feature
gives robust results without needing to loosen the tolerance.


Initially we consider the same basic models as in~\cite{Funakoshi};
a quadratic potential
\begin{align}\label{eq:quadratic_potential}
    \quadpot = \frac{1}{2}m^2\phi^2.
\end{align}
with a canonical kinetic term,
and a non-canonical model, with a DBI
kinetic term
\begin{align}\label{eq:dbi_action}
    S_{DBI}=\int d^4x\sqrt{-g}\left(-\frac{1}{f(\phi)}\left(\left(1+f(\phi)\partial_\mu\phi\partial^\mu\phi\right)^{\frac{1}{2}}-1\right)-V(\phi)\right),
\end{align}
with
\begin{align}\label{eq:dbi_warp}
    f(\phi)=\frac{\lambda_{DBI}}{\phi^4},\qquad
    V(\phi)=V_0-\frac{1}{2}m^2\phi^2,\qquad
    m=\sqrt{\beta_{IR}}H.
\end{align}

For our more stringent validation tests we work with feature model scenarios
based on the above base models.
To explore non-Gaussianity coming from sharp features we include
a kink
\begin{align}\label{eq:kink_potential}
    V(\phi) = \quadpot\left(1-c\tanh\left(\frac{\phi_f-\phi}{d}\right)\right).
\end{align}
To explore non-Gaussianity from deeper in the horizon we imprint
extended resonant features on the basic potential
\begin{align}\label{eq:resonant_potential}
    V(\phi) = \quadpot\left(1+bf\sin\left(\frac{\phi}{f}\right)\right).
\end{align}
For more details on these models, see~\cite{chen_easther_lim_2}.
To express the bispectrum results more compactly we use the symmetric polynomial notation employed in~\cite{FergShell_2}:
\begin{align}\label{shape_notation}
\begin{split}
    K_p &= \sum_{i=1,2,3} k_i^p, \\
    K_{pq} &= \frac{1}{\Delta_{pq}}\sum_{i\neq j} k_i^p k_j^q,   \\
    K_{prs} &= \frac{1}{\Delta_{prs}}\sum_{i\neq j\neq l} k_i^p k_j^r k_l^s,
\end{split}
\end{align}
where $\Delta_{pq}$ is $2$ if $p=q$, $1$ otherwise
and $\Delta_{prs}$ is $6$ if $p=r=s$, $2$ if $p=r\neq s$ (and permutations),
and $1$ if $p,r,s$ are all distinct.
With a canonical kinetic term, the slow-roll result for the shape is:
\begin{align}\label{malda_shape}
    S^{Malda}(k_1,k_2,k_3) &= A^{Malda} \left( (3\varepsilon-2\eta)\frac{K_3}{K_{111}}+\varepsilon \left(K_{12}+8\frac{K_{22}}{K}\right) \right),\\
    A^{Malda} &= -\frac{1}{32}\frac{H^4}{12\varepsilon^2}.
\end{align}
with $\eta=2\varepsilon$ for~\eqref{eq:quadratic_potential}.
At the primordial level, this is well approximated by the separable local template
\begin{align}\label{local_shape}
S^{local}(k_1,k_2,k_3) = \frac{k_1^2}{k_2k_3}+\frac{k_2^2}{k_3k_1}+\frac{k_3^2}{k_1k_2} = \frac{K_3}{6\,K_{111}}.
\end{align}
However, the amplitude of this shape is expected to be tiny,
and the dominant contributions (the squeezed configurations) are expected
to have no observable effect~\cite{Cabass_2016}.
The local template is in fact used to test for multi-field effects~\cite{Planck_NG_2015}.
For the featureless DBI scenario, the shape function is~\cite{dbi_in_the_sky}:
\begin{align}\label{dbi_shape}
    S^{DBI}(k_1,k_2,k_3) &= A^{DBI}\frac{K_5+2K_{14}-3K_{23}+2K_{113}-8K_{122}}{K_{111}K^2},\\
    A^{DBI} &= -\frac{1}{32}\frac{H^4}{12\varepsilon^2}\left(\frac{1}{c_s^2}-1\right),
\end{align}
to leading order in slow-roll.
Any constraint on the magnitude $A^{DBI}$ can be translated into one 
on the effective sound speed which from \planck~has a lower limit $c_s^{DBI} \geq 0.087$
at $95\%$ significance~\cite{Planck_NG_2015}.
The shape~\eqref{dbi_shape} can be approximated by the separable equilateral template
\begin{align}\label{equil_shape}
    S^{equil}(k_1,k_2,k_3) = \frac{(k_2+k_3-k_1)(k_3+k_1-k_2)(k_1+k_2-k_3)}{k_1k_2k_3}.
\end{align}
These templates can be modified to be more physically realistic by including
scaling consistent with the spectral index $n_s$~\cite{Planck_NG_2015}.
For example, we can add some scale dependence to the DBI model in a reasonable first approximation by including a prefactor. We define the product scaling template
\begin{align}\label{dbi_prod_shape}
    S^{DBI-n_s}(k_1,k_2,k_3) &= {\left(\frac{k_1k_2k_3}{k^3_\star}\right)}^{\frac{n_{NG}}{3}}S^{DBI}(k_1,k_2,k_3)
\end{align}
and the sum scaling template
\begin{align}\label{dbi_sum_shape}
    S^{DBI-n_s}(k_1,k_2,k_3) &= {\left(\frac{k_1+k_2+k_3}{3k_\star}\right)}^{n_{NG}}S^{DBI}(k_1,k_2,k_3)
\end{align}
with $n_{NG}=2(-2\varepsilon-\varepsilon_s-\eta)-2\varepsilon_s=2(n_s-1)-2\varepsilon_s$.

We now turn to feature templates.
The result of adding a feature of the form~\eqref{eq:kink_potential}
is to add oscillatory features of the form
\begin{align}\label{cos_shape}
    S^{\cos}(k_1,k_2,k_3) = \cos(w(k_1+k_2+k_3))
\end{align}
though more realistically there is some phase, shape dependence and a modulating envelope,
as detailed in~\cite{adshead}.
The result of adding a resonant feature of the form~\eqref{eq:resonant_potential}
is to generate logarithmic oscillatory features in the shape function of the form
\begin{align}\label{ln_cos_shape}
    S^{\ln-\cos}(k_1,k_2,k_3) = \cos(w\ln(k_1+k_2+k_3)).
\end{align}
With a non-canonical kinetic term, this can also
cause out-of-phase oscillations in the folded limit as well as a modulating shape,
see~\cite{chen_folded_resonant}.


Much success has been had in constraining non-Gaussianity
in the CMB using separable approximations to these approximate templates.
Other methods target oscillations~\cite{reso_estimator}, by expanding the shape function
in $k_1+k_2+k_3$, thus limiting their ability to capture shapes whose
phase varies across the tetrapyd.
Our motivation in this work for directly calculating the primordial
bispectrum in a separable form is to build towards
a pipeline to constrain a broader section of the model space,
removing these layers of approximations,
though these standard results provide useful validation tests.
    \subsection{Shapes from features during inflation}
    Explicit details of how resonance and features generate large NG.
\section{Previous work on in-in separability.}
    In~\cite{Funakoshi} it was pointed out that one can compute using the
tree-level in-in formalism in such a way as to preserve its intrinsic
separability. In addition to making this point,~\cite{Funakoshi} lays
out some of the basic structure of an implementation of that computation,
and validates the method on simple, featureless scenarios.
This work built on the philosophy of~\cite{FergShell_1,FergShell_2,FergShell_3}
in which a formalism was developed to
leverage the tractability of separable CMB bispectrum estimation
for generic primordial bispectra, by expanding them in a separable basis.
The results of these methods (not using the work of~\cite{Funakoshi})
are constraints on the parameters of certain inflation models through approximate
phenomenological templates.
These constraints can be found in~\cite{Planck_NG_2015, Planck_NG_2018}.
The idea of~\cite{Funakoshi} is an extension of that philosophy to the primordial level,
and our work is in implementing that idea.
In~\cite{FergShell_1,FergShell_2,FergShell_3} an orthogonal basis on the tetrapyd was used,
removing the need to fit non-physical configurations.
One of the main differences between that work and this
is that we cannot use this basis here without sacrificing the
in-in separability we are trying to preserve.

In this work we explore the details of this calculation in much greater detail
than was considered in~\cite{Funakoshi}.
We restructure the methods, improving on the work of~\cite{Funakoshi} in terms
of flexibility of basis choice and efficiency of the calculation.
We also detail a particular set of basis functions that improves upon those described
in~\cite{Funakoshi} in its rate of convergence, its transparency,
and its flexibility.
We do this without sacrificing orthogonality.
This is detailed in chapter~\ref{chapter:decomp}.
Our improvements over the methods sketched in~\cite{Funakoshi} allow us to validate
on non-trivial bispectra for the first time, including sharp deviations from slow-roll, which we present in
section~\ref{sec:validation}.
We quote our results in terms of a measure that is
easier to interpret than the correlation defined in~\cite{Funakoshi},
and that includes the magnitude as well as the shape information
on the full tetrapyd.
This is discussed in section~\ref{sec:inner_product}.
    \subsection{Comparison to the present work}
    Summary of the achievements and limitations of~\cite{Funakoshi}, how I went beyond them.
    \section{Configuration-by-configuration codes}
    Previous work on the numerical calculations of inflationary
non-Gaussianity include the BINGO code~\cite{BINGO},
Chen et al~\cite{chen_easther_lim_1,chen_easther_lim_2},
the work of Horner et al~\cite{horner_methods,horner_ng,horner_cs}
and the Transport Method~\cite{transport_main,transport_pytransport,transport_pytransport_2,transport_curved_3_point}.
All but the last directly apply the tree-level in-in formalism $k$-configuration by $k$-configuration for a given model;
they integrate a product of three mode functions and a background-dependent term from the interaction Hamiltonian, of form similar to~\eqref{inin_sep}.
The eventual result is a grid of points representing the primordial bispectrum.


The most advanced publicly released code for the calculation of inflationary perturbations
is based on the Transport Method.
Like the previously mentioned work it calculates the bispectrum $k$-configuration by $k$-configuration.
However the method is different in its details.
Instead of performing integrals,
a set of coupled ODEs is set up and solved.
The power spectra and bispectra themselves are evolved, their time derivatives calculated by
differentiating the in-in formalism.\footnote{See~\ref{appendix_modal_transport}
    for details on applying this strategy to our own modal coefficients.
    }
The publicly released code is very sophisticated,
able to deal with multiple fields in curved field spaces,
recently being used to explore the bispectra resulting from
sidetracked inflation~\cite{RP_1}.
    \subsection{Usage in recent works}
    Sidetracked inflation, etc\ldots~\cite{Fumagalli_2019} using the transport method~\cite{transport_pytransport_2}.
    \subsection{Limitations}
    However despite the differences, all configuration-by-configuration methods face the same problems:
firstly, that calculating enough points in the bispectrum to ensure that
the whole picture has been captured is expensive, especially for non-trivial features.
Even once that has been achieved, what is obtained is a grid of points
which must be processed further to be usefully compared to observation.
Secondly, they must carefully implement some variation
of the $i\eps$ prescription without affecting the numerical results.
In~\cite{transport_main} this is achieved in the initial conditions for the bispectra;
other methods impose some non-trivial cutoff at early times.

    \newpage
    \section{The squeezed limit consistency condition}
    Could this be better placed somewhere?
\newpage
    \section{Review forecasts for future surveys}
    E.g.~\cite{Sohn_2019}. Talk about Astro2020 papers here?~\cite{astro2020_png, astro2020_features}
    \newpage
    \section{Review estimators, KSW, separability}
    E.g.~\cite{Smith_2011, Komatsu_2005}.


    The temperature anisotropies of the $\cmb$ are decomposed in spherical
    harmonics as
    \begin{align}
        \frac{\Delta T}{T}(\hat{\mathbf{n}}) &= \sum_{lm} a_{lm}Y_{lm}(\hat{\mathbf{n}})
    \end{align}
    and the power spectrum is
    \begin{align}
        C_l &= \sum_m a_{lm} a_{l-m}.
    \end{align}


    The bispectrum derived from theory is denoted
    \begin{align}
        \left<\triplea\right>
    \end{align}
    whereas the bispectrum measured from the $\cmb$ is
    \begin{align}
        B^{l_1l_2l_3}_{m_1m_2m_3} &= \tripleaobs.
    \end{align}
    We also have
    \begin{align}
        B_{l_1l_2l_3} = \sum_{m_i} {{l_1~~l_2~~l_3} \choose {m_1~m_2~m_3}} B^{l_1l_2l_3}_{m_1m_2m_3}.
    \end{align}


    The standard method for estimating the $\cmb$ bispectrum is to calculate the least squares
    fit between the theory bispectrum and the observed bispectrum. That is, we find $\lambda$
    such that the following expression is minimised:
    \begin{align}
        \sum_{l_i,m_i}{\left(\lambda\frac{\left<\triplea\right>}{\sqrt{C_{l_1}C_{l_2}C_{l_3}}}
                - \frac{\tripleaobs}{\sqrt{C_{l_1}C_{l_2}C_{l_3}}}\right)}^2.
    \end{align}
    The solution to this is simply
    \begin{align}
        \lambda = \frac{\frac{\left<\triplea\right>}{\sqrt{C_{l_1}C_{l_2}C_{l_3}}}\cdot\frac{\tripleaobs}{\sqrt{C_{l_1}C_{l_2}C_{l_3}}}}{\left|\frac{\left<\triplea\right>}{\sqrt{C_{l_1}C_{l_2}C_{l_3}}}\right|^2}
    \end{align}
    where $\cdot$ denotes summation over $l_i$ and $m_i$.
    Defining
    \begin{align}
        N = \left|\frac{\left<\triplea\right>}{\sqrt{C_{l_1}C_{l_2}C_{l_3}}}\right|^2
    \end{align}
    we can rewrite this in the usual way (with $\lambda$ identified as the result of the estimator $\mathcal{E}$)
    \begin{align}
        \mathcal{E} = \frac{1}{N}\sum_{l_i,m_i}\frac{\left<\triplea\right>\tripleaobs}{C_{l_1}C_{l_2}C_{l_3}}.
    \end{align}

    \section{Separable approximations to non-separable templates}
    Separability is required, so usually one approximates non-separable templates
    by separable ones.
\newpage
    Focus on equilateral to DBI.
    Quantitative comparison of equilateral template and DBI shape,
    with and without scaling.
\newpage
    \section{Estimating the bispectrum, complexity}
    The bispectrum, like the power spectrum, is a quantity that describes
the statistical distribution of which our universe is only one realisation.
We use this one sky we have access to to estimate the amplitude of
particular bispectrum templates,
and use these estimates to constrain inflationary physics; 
see~\cite{astro2020_features,astro2020_png} for recent reviews.
There are two parts to the pipeline of bispectrum estimation.
Firstly, calculating the primordial bispectrum at the end of some inflation scenario,
and then calculating the effect this bispectrum
has on some appropriate observable today.
One well-developed example is
the bispectrum of temperature fluctuations in the CMB, which uses transfer functions
to evolve and project the primordial bispectrum onto our sky.
In principle, this is the same process as power spectrum estimation.
However, for the bispectrum the computational challenge is far greater,
requiring both compute-intensive and large in-memory components.


As a result of this complexity, this second step is computationally impractical for generic primordial bispectra.
Progress can be made by finding an approximation to the primordial shape
that is separable, and using this simplification
to make the calculation tractable
through the KSW estimator~\cite{Komatsu_2005, Munchmeyer_2014}.
For example, one may find that a particular inflation scenario generates
a primordial bispectrum with a high correlation with some standard shape,
then look at how well that standard shape is constrained by the CMB.
The modal decomposition method of~\cite{FergShell_1,FergShell_2,FergShell_3}
leveraged these simplifications in a more structured way
for generic bispectra, broadening the range of constrained models.


The measure of non-Gaussianity in the CMB that is
most usually quoted is $\fnl$, referring to $\fnllocal$.
This number describes how well a particular template, the local template,
describes the correlations in the CMB;
this template is used as a proxy for the class of inflation models that produce similar bispectra.
Similar quantities for the equilateral and orthogonal templates are also
commonly quoted.
In addition to broadening the range of constrained models through increases in efficiency,
the modal decomposition method of~\cite{FergShell_1,FergShell_2,FergShell_3}
allows to go beyond this paradigm, efficiently constraining inflationary bispectra in the CMB using
all of the shape information; essentially constraining an $\fnl$
specific to a given bispectrum. This bypasses the approximation step at the level of the templates,
of finding a separable approximation to the primordial bispectrum.
In this work, our numerical methods remove the need for some of the approximations
made before this, during inflation, directly linking the parameters of the inflation scenario
with the relevant observable.
In addition to this improvement in accuracy, calculating the modal decomposition
directly from the model of inflation is far more efficient 
than numerically calculating the bispectrum configuration by configuration.


If the shape function~\eqref{shapefn} has the form:
\begin{equation}\label{sepXYZ}
S(k_1,k_2,k_3) = X(k_1)Y(k_2)Z(k_3),
\end{equation}
or can be expressed as a sum of such terms,
it is called separable.
The link between the separability of the primordial bispectrum
and the reduced CMB bispectrum can be seen from the following expression:
\begin{equation}
\label{eq:reduced_cmb}
b^{X_1X_2X_3}_{l_1l_2l_3} = \left(\frac{2}{\pi}\right)^3\int_{0}^{\infty}drr^2\int_{\mathcal{V}_k}d^3k (k_1k_2k_3)^2 B_{\Phi}(k_1,k_2,k_3)\prod_{i=1}^{3}\left[j_{l_i}(k_ir)\Delta^{X_i}_{l_i}(k_i)\right],
\end{equation}
where we also see that if the primordial bispectrum is separable
then the overall dimension
of the calculation can be reduced from seven to five, 
since the spherical Bessel functions $j_{l_i}$ and the
transfer functions $\Delta_{l_i}$ already appear in a separable way.
This property can also be used to
efficiently generate non-Gaussian initial conditions
for simulations~\cite{Scoccimarro_2012}.


The numbers $\fnl^F$ are useful summary parameters.
From the data-side, they represent the result of
a complex and intensive process
of estimating the amplitude of the template $F$,
given some data. From the theory-side, one
can use them to take an inflation scenario and compare it
to that data, if one can find a standard template
with a high correlation with the shape resulting
from that scenario.
However, despite its usefulness, this paradigm does
have drawbacks. It acts as an information bottleneck,
losing some constraining power when one approximates
the real shape function by some standard template.
In particular, if one is interested in a feature model,
it may be be difficult to see how constraints on existing
features can be applied.
\newpage
    \section{Modal methods, constraints from Planck}
    Leveraging the separable benefits in a broader set of models through expansion.
\newpage
    \section{Wuhyun's work as development of Modal methods}
    Contrast Wuhyun's work (our pipeline) to previous modal methods.
    What Wuhyun does is purely frequentist, not Bayesian. It is essentially linear regression, so can do Fisher forecast to determine
    the best possible variance for an estimator.
    

